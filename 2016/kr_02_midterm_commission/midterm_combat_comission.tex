\documentclass[12pt]{article}\usepackage[]{graphicx}\usepackage[svgnames]{xcolor}
%% maxwidth is the original width if it is less than linewidth
%% otherwise use linewidth (to make sure the graphics do not exceed the margin)
\makeatletter
\def\maxwidth{ %
  \ifdim\Gin@nat@width>\linewidth
    \linewidth
  \else
    \Gin@nat@width
  \fi
}
\makeatother

\definecolor{fgcolor}{rgb}{0.345, 0.345, 0.345}
\newcommand{\hlnum}[1]{\textcolor[rgb]{0.686,0.059,0.569}{#1}}%
\newcommand{\hlstr}[1]{\textcolor[rgb]{0.192,0.494,0.8}{#1}}%
\newcommand{\hlcom}[1]{\textcolor[rgb]{0.678,0.584,0.686}{\textit{#1}}}%
\newcommand{\hlopt}[1]{\textcolor[rgb]{0,0,0}{#1}}%
\newcommand{\hlstd}[1]{\textcolor[rgb]{0.345,0.345,0.345}{#1}}%
\newcommand{\hlkwa}[1]{\textcolor[rgb]{0.161,0.373,0.58}{\textbf{#1}}}%
\newcommand{\hlkwb}[1]{\textcolor[rgb]{0.69,0.353,0.396}{#1}}%
\newcommand{\hlkwc}[1]{\textcolor[rgb]{0.333,0.667,0.333}{#1}}%
\newcommand{\hlkwd}[1]{\textcolor[rgb]{0.737,0.353,0.396}{\textbf{#1}}}%
\let\hlipl\hlkwb

\usepackage{framed}
\makeatletter
\newenvironment{kframe}{%
 \def\at@end@of@kframe{}%
 \ifinner\ifhmode%
  \def\at@end@of@kframe{\end{minipage}}%
  \begin{minipage}{\columnwidth}%
 \fi\fi%
 \def\FrameCommand##1{\hskip\@totalleftmargin \hskip-\fboxsep
 \colorbox{shadecolor}{##1}\hskip-\fboxsep
     % There is no \\@totalrightmargin, so:
     \hskip-\linewidth \hskip-\@totalleftmargin \hskip\columnwidth}%
 \MakeFramed {\advance\hsize-\width
   \@totalleftmargin\z@ \linewidth\hsize
   \@setminipage}}%
 {\par\unskip\endMakeFramed%
 \at@end@of@kframe}
\makeatother

\definecolor{shadecolor}{rgb}{.97, .97, .97}
\definecolor{messagecolor}{rgb}{0, 0, 0}
\definecolor{warningcolor}{rgb}{1, 0, 1}
\definecolor{errorcolor}{rgb}{1, 0, 0}
\newenvironment{knitrout}{}{} % an empty environment to be redefined in TeX

\usepackage{alltt}



\usepackage[top=3cm, left=2cm, right=2cm]{geometry} % размер текста на странице

\usepackage[box, % запрет на перенос вопросов
%nopage,
insidebox, % ставим буквы в квадратики
separateanswersheet, % добавляем бланк ответов
nowatermark, % отсутствие надписи "Черновик"
% indivanswers,  % показываем верные ответы
%answers,
lang=RU,
nopage, % убираем оформление страницы (идентификаторы для распознавания)
completemulti]{automultiplechoice}

\usepackage{tikz} % картинки в tikz
\usepackage{microtype} % свешивание пунктуации

\usepackage{dcolumn} % для разделения по десятичной точке (для функции mtable)
\usepackage{comment} % для многострочных комментариев

\usepackage{array} % для столбцов фиксированной ширины

\usepackage{indentfirst} % отступ в первом параграфе

\usepackage{sectsty} % для центрирования названий частей
\allsectionsfont{\centering}

\usepackage{amsmath, amsfonts} % куча стандартных математических плюшек


\usepackage{multicol} % текст в несколько колонок

\usepackage{lastpage} % чтобы узнать номер последней страницы

\usepackage{enumitem} % дополнительные плюшки для списков
%  например \begin{enumerate}[resume] позволяет продолжить нумерацию в новом списке










\usepackage{fancyhdr} % весёлые колонтитулы
\pagestyle{fancy}
\lhead{Эконометрика, пересдача}
\chead{}
\rhead{15.02.2017}
\lfoot{}
\cfoot{}
\rfoot{\thepage/\pageref{LastPage}}
\renewcommand{\headrulewidth}{0.4pt}
\renewcommand{\footrulewidth}{0.4pt}



\usepackage{todonotes} % для вставки в документ заметок о том, что осталось сделать
% \todo{Здесь надо коэффициенты исправить}
% \missingfigure{Здесь будет Последний день Помпеи}
% \listoftodos — печатает все поставленные \todo'шки


% более красивые таблицы
\usepackage{booktabs}
% заповеди из докупентации:
% 1. Не используйте вертикальные линни
% 2. Не используйте двойные линии
% 3. Единицы измерения - в шапку таблицы
% 4. Не сокращайте .1 вместо 0.1
% 5. Повторяющееся значение повторяйте, а не говорите "то же"



\usepackage{fontspec}
\usepackage{polyglossia}

\setmainlanguage{russian}
\setotherlanguages{english}

% download "Linux Libertine" fonts:
% http://www.linuxlibertine.org/index.php?id=91&L=1
\setmainfont{Linux Libertine O} % or Helvetica, Arial, Cambria
% why do we need \newfontfamily:
% http://tex.stackexchange.com/questions/91507/
\newfontfamily{\cyrillicfonttt}{Linux Libertine O}

\AddEnumerateCounter{\asbuk}{\russian@alph}{щ} % для списков с русскими буквами


%% эконометрические сокращения
\DeclareMathOperator{\plim}{plim}
\DeclareMathOperator{\Cov}{Cov}
\DeclareMathOperator{\Corr}{Corr}
\DeclareMathOperator{\Var}{Var}
\DeclareMathOperator{\E}{E}
\def \hb{\hat{\beta}}
\def \hs{\hat{\sigma}}
\def \htheta{\hat{\theta}}
\def \s{\sigma}
\def \hy{\hat{y}}
\def \hY{\hat{Y}}
\def \v1{\vec{1}}
\def \e{\varepsilon}
\def \he{\hat{\e}}
\def \z{z}

\def \sVar{\widehat{\Var}}
\def \sCorr{\widehat{\Corr}}
\def \sCov{\widehat{\Cov}}



\def \hVar{\widehat{\Var}}
\def \hCorr{\widehat{\Corr}}
\def \hCov{\widehat{\Cov}}
\def \cN{\mathcal{N}}


\AddEnumerateCounter{\asbuk}{\russian@alph}{щ} % для списков с русскими буквами
\setlist[enumerate, 2]{label=\asbuk*),ref=\asbuk*}
\IfFileExists{upquote.sty}{\usepackage{upquote}}{}
\begin{document}


%<<child="retake2_test_bank.Rnw">>=
%@
\element{comission}{ % в фигурных скобках название группы вопросов
 %\AMCnoCompleteMulti
  \begin{questionmult}{pr1} % тип вопроса (questionmult — множественный выбор) и в фигурных — номер вопроса

    В множественной регрессии с двумя регрессорами выборочные корреляции между зависимой переменной и регрессорами составили: $\widehat{\Corr}(Y, X_1) = 0.7$, $\widehat{\Corr}(Y, X_2) = 0.2$. Тогда $R^2$ будет равен

 \begin{multicols}{3} % располагаем ответы в 3 колонки
   \begin{choices} % опция [o] не рандомизирует порядок ответов
      \correctchoice{Не хватает данных для ответа}
      \wrongchoice{0.49}
      \wrongchoice{0.04}
      \wrongchoice{0.81}
      \wrongchoice{0.25}
      \wrongchoice{0.9}
   \end{choices}
  \end{multicols}
  \end{questionmult}
}



\element{comission}{ % в фигурных скобках название группы вопросов
 %\AMCnoCompleteMulti
  \begin{questionmult}{pr2} % тип вопроса (questionmult — множественный выбор) и в фигурных — номер вопроса

    Исследователь Феофан оценил регрессию $Y$ на $Z$ и получил, что $\hat{Y_i} = 10 + 2Z_i$. После этого, он оценил регрессию $Y$ на $Z$ и новую переменную $X$. Известна выборочная ковариация, $\widehat{\Cov}(Y,X) = 0$. Выберете верное утверждение:

 \begin{multicols}{2} % располагаем ответы в 3 колонки
   \begin{choices} % опция [o] не рандомизирует порядок ответов
      \correctchoice{$R^2$ не снизился по сравнению с исходной моделью}
      \wrongchoice{Коэффициент при переменной $Z$ не изменился}
      \wrongchoice{Коэффициент при переменной $X$ равен нулю}
      \wrongchoice{$R^2_{adj}$ вырос и стал равен 1}
      \wrongchoice{Коэффициент при $Z$ стал незначимым}
      \wrongchoice{Коэффициент при $Z$ остался значимым}
   \end{choices}
  \end{multicols}
  \end{questionmult}
}




\element{comission}{ % в фигурных скобках название группы вопросов
 %\AMCnoCompleteMulti
  \begin{questionmult}{pr3} % тип вопроса (questionmult — множественный выбор) и в фигурных — номер вопроса

    Исследовательница Алевтина изучает зависимость размера порции в мишленовском ресторане от его звёздности, $star$ (от 1 до 3), и уровня цен, $price$. Она оценила модель вида $size_i = \beta_1 + \beta_2 star2_i + \beta_3 star3_i + \beta_4 price_i + u_i$, где $star1, star2, star3$ - дамми-переменные, равные 1 для ресторанов с соответствующим числом звезд, и 0 иначе. Алевтина считает, что размер порции уменьшается вдвое с каждой дополнительной звездой. Какую гипотезу ей нужно проверить?

 \begin{multicols}{2} % располагаем ответы в 3 колонки
   \begin{choices} % опция [o] не рандомизирует порядок ответов
      \correctchoice{ $H_0 : \beta_1 = 2\beta_2 = 4\beta_3$ }
      \wrongchoice{ $H_0 : \beta_3 = 4, \beta_2 = 2, \beta_1 = 1$ }
      \wrongchoice{ $H_0 : 4\beta_1 = 2\beta_2 = \beta_3$ }
      \wrongchoice{ $H_0 : \beta_1 < \beta_2 < \beta_3$ }
      \wrongchoice{ $H_0 : \beta_1 = 2\beta_2 = 3\beta_3$ }
      \wrongchoice{ $H_0 : 3\beta_1 = 2\beta_2 = \beta_3$ }
   \end{choices}
  \end{multicols}
  \end{questionmult}
}




\element{comission}{ % в фигурных скобках название группы вопросов
 %\AMCnoCompleteMulti
  \begin{questionmult}{pr4} % тип вопроса (questionmult — множественный выбор) и в фигурных — номер вопроса

    Исследовательница Алевтина вновь изучает зависимость размера порции в мишленовском ресторане от его звёздности, $star$ (от 1 до 3), и уровня цен, $price$. Она оценила модель вида $size_i = \beta_1 + \beta_2 star2_i + \beta_3 star3_i + \beta_4 price_i + u_i$, где $star1, star2, star3$ - дамми-переменные, равные 1 для ресторанов с соответствующим числом звезд, и 0 иначе. У Алевтины есть $n$ наблюдений. С помощью какой статистики она будет проверять гипотезу об отсутствии влияния звёздности на размер порции?

 \begin{multicols}{3} % располагаем ответы в 3 колонки
   \begin{choices} % опция [o] не рандомизирует порядок ответов
      \correctchoice{ $F_{2, n-4}$ }
      \wrongchoice{ $t_{n-4}$ }
      \wrongchoice{ $t_{n-3}$ }
      \wrongchoice{ $F_{3, n-4}$ }
      \wrongchoice{ $F_{2, n-3}$ }
      \wrongchoice{ $F_{3, n-3}$ }
   \end{choices}
  \end{multicols}
  \end{questionmult}
}



\element{comission}{ % в фигурных скобках название группы вопросов
 %\AMCnoCompleteMulti
  \begin{questionmult}{pr5} % тип вопроса (questionmult — множественный выбор) и в фигурных — номер вопроса

    Исследователь Валериан оценил модель зависимости дохода человека, $income$, от его возраста, $age$, и получил следующую зависимость $\widehat{income}_i = 150 + 900 age_i - 10 age_i^2$. Все коэффициенты значимы на 5\% уровне значимости. В каком возрасте доход достигает максимума?

 \begin{multicols}{2} % располагаем ответы в 3 колонки
   \begin{choices} % опция [o] не рандомизирует порядок ответов
      \correctchoice{ 45 лет }
      \wrongchoice{ 90 лет }
      \wrongchoice{ 30 лет }
      \wrongchoice{ Недостаточно данных }
      \wrongchoice{ Доход возрастает с ростом возраста, у функции нет максимума }
      \wrongchoice{ Доход убывает с ростом возраста, у функции нет максимума }
   \end{choices}
  \end{multicols}
  \end{questionmult}
}



\element{comission}{ % в фигурных скобках название группы вопросов
 %\AMCnoCompleteMulti
  \begin{questionmult}{pr6} % тип вопроса (questionmult — множественный выбор) и в фигурных — номер вопроса

    В регрессии с четырьмя регрессорами, оцененной по 21 наблюдению, оказалось, что $TSS = 250$, $R^2_{adj} = 0.75$. Тогда $R^2$ равен

 \begin{multicols}{3} % располагаем ответы в 3 колонки
   \begin{choices} % опция [o] не рандомизирует порядок ответов
      \correctchoice{ 0.8 }
      \wrongchoice{ 0.2 }
      \wrongchoice{ 0.6 }
      \wrongchoice{ 0.96 }
      \wrongchoice{ 0.7 }
      \wrongchoice{ 0.3 }
   \end{choices}
  \end{multicols}
  \end{questionmult}
}



\element{comission}{ % в фигурных скобках название группы вопросов
 %\AMCnoCompleteMulti
  \begin{questionmult}{pr7} % тип вопроса (questionmult — множественный выбор) и в фигурных — номер вопроса

    Регрессия с тремя регрессорами оценена по 100 наблюдениям. Какая из этих гипотез НЕ может быть проверена при помощи статистики, имеющей $F_{2,96}$ распределение?

 \begin{multicols}{3} % располагаем ответы в 3 колонки
   \begin{choices} % опция [o] не рандомизирует порядок ответов
      \correctchoice{ $H_0: \beta_2 = 2 \beta_3$ }
      \wrongchoice{ $H_0: \beta_2 = \beta_3 = 0$ }
      \wrongchoice{ $H_0: \beta_1 = 1; \beta_3 = 5$ }
      \wrongchoice{ $H_0: \beta_2 = 3 \beta_3 = 5$ }
      \wrongchoice{ $H_0: \beta_1 + \beta_2 = 10; \beta_3 = 1$ }
      \wrongchoice{ $H_0: \beta_1 = \beta_2 = \beta_3$ }
   \end{choices}
  \end{multicols}
  \end{questionmult}
}



\element{comission}{ % в фигурных скобках название группы вопросов
 %\AMCnoCompleteMulti
  \begin{questionmult}{pr8} % тип вопроса (questionmult — множественный выбор) и в фигурных — номер вопроса

    Исследователь выполнил второй шаг в PE-тесте МакКиннона. В регрессии $\ln Y_i$ на исходные регрессоры и $Z_i = \hat Y_i - \exp(\widehat{\ln Y_i})$ коэффициент при $Z_i$ оказался значимым. А в регрессии $Y_i$ на исходные регрессоры и $W_i = \ln \hat Y_i - \widehat{\ln Y_i}$ коэффициент при $W_i$ оказался незначимым. Из результатов следует сделать вывод, что

 \begin{multicols}{2} % располагаем ответы в 3 колонки
   \begin{choices} % опция [o] не рандомизирует порядок ответов
      \correctchoice{следует предпочесть линейную модель}
      \wrongchoice{следует предпочесть логарифмическую модель}
      \wrongchoice{следует предпочесть полулогарифмическую модель}
      \wrongchoice{тесты противоречат друг другу, ни одна из моделей не предпочитается}
      \wrongchoice{в исходной модели пропущен регрессор $Z_i$}
      \wrongchoice{в исходной модели пропущен регрессор $W_i$}
   \end{choices}
  \end{multicols}
  \end{questionmult}
}



\element{comission}{ % в фигурных скобках название группы вопросов
 %\AMCnoCompleteMulti
  \begin{questionmult}{pr9} % тип вопроса (questionmult — множественный выбор) и в фигурных — номер вопроса

    Истинной является модель $Y_i = \beta_1 + \beta_2 X_i + u_i$. Глафира оценивает две регрессии: $\hat Y_i = \hb_1 + \hb_2X_i$ и  $\hat Y_i = \hat \gamma_1 + \hat \gamma_2 X_i + \hat \gamma_3 Z_i$ с помощью МНК. Известна выборочная корреляция $\widehat{\Corr}(X_i, Z_i) = -0.2$. Тогда оценка $\hat \gamma_2$ является

 \begin{multicols}{1} % располагаем ответы в 3 колонки
   \begin{choices} % опция [o] не рандомизирует порядок ответов
      \correctchoice{состоятельной, но неэффективной оценкой для $\beta_2$}
      \wrongchoice{несостоятельной и неэффективной оценкой для $\beta_2$}
      \wrongchoice{несостоятельной, но эффективной оценкой для $\beta_2$}
      \wrongchoice{состоятельной, но смещенной на $-0.2 \hat \gamma_3$ относительно $\beta_2$ оценкой}
      \wrongchoice{состоятельной, но смещенной на $-0.2$ относительно $\beta_2$ оценкой}
      \wrongchoice{смещённой на $-0.2 \hat \gamma_3$ относительно $\beta_2$, но эффективной оценкой}
   \end{choices}
  \end{multicols}
  \end{questionmult}
}



\element{comission}{ % в фигурных скобках название группы вопросов
 %\AMCnoCompleteMulti
  \begin{questionmult}{pr10} % тип вопроса (questionmult — множественный выбор) и в фигурных — номер вопроса

    Если для регрессора используется преобразование Бокса-Кокса с параметром $\theta = -1$, а для зависимой переменной — с параметром $\lambda = 0$, то регрессионное уравнение представимо в виде

 \begin{multicols}{2} % располагаем ответы в 3 колонки
   \begin{choices} % опция [o] не рандомизирует порядок ответов
      \correctchoice{ $\ln Y_i = \beta_1 + \beta_2 \frac{1}{X_i} + u_i$ }
      \wrongchoice{ $Y_i = \beta_1 + \beta_2 \frac{1}{X_i} + u_i$ }
      \wrongchoice{ $\ln Y_i = \beta_1 + \beta_2 X_i + u_i$ }
      \wrongchoice{ $Y_i = \beta_1 + \beta_2 X_i + u_i$ }
      \wrongchoice{ $\ln Y_i = \beta_1 + \beta_2 (X_i - 1) + u_i$ }
      \wrongchoice{ $\ln Y_i = \beta_1 + \beta_2 X^2_i + u_i$ }
   \end{choices}
  \end{multicols}
  \end{questionmult}
}



\section*{Часть 1. Тест.}

\onecopy{1}{

\cleargroup{combat}
\copygroup[10]{comission}{combat}
\shufflegroup{combat}
\insertgroup{combat}

}

\section*{Часть 2. Задачи.}


\begin{enumerate}

\item Пусть $y_i = \beta_1 + \beta_2 x_i + \e_i$ и $i = 1, \dots, 5$ — классическая регрессионная модель. Также имеются следующие данные: $\sum_{i=1}^5 y_i^2 = 55, \sum_{i=1}^5 x_i^2 = 3, \sum_{i=1}^5 x_iy_i = 12, \sum_{i=1}^5 y_i = 15, \sum_{i=1}^5 x_i = 3$.

\begin{enumerate}
\item Найдите $\hat{\beta_1}$, $\hat{\beta_2}$
\item Найдите $TSS$, $ESS$, $RSS$, $R^2$, $\hs^2$
\end{enumerate}


\item Исследовательница Глафира изучает зависимость спроса на молоко от цены молока и дохода семьи. В её распоряжении есть следующие переменные:

\begin{itemize}
\item $price$ — цена молока в рублях за литр
\item $income$ — ежемесячный доход семьи в тысячах рублей
\item $milk$ — расходы семьи на молоко за последние семь дней в рублях
\end{itemize}

В данных указано, проживает ли семья в сельской или городской местности. Поэтому Глафира оценила три регрессии: (All) — по всем данным, (Urban) — по городским семьям, (Rural) — по сельским семьям.

%%%%%%%%%%%%%%%%%%%%%%%%%%%%%%%%%%%%%%%%%%%%%%%%%%%%%%%%%%%%%%%%%%%%%%%%%%%%%%%%%%%%%%%
%
% Calls:
% (All):  lm(formula = milk ~ income + price, data = milk_demand) 
% (Urban):  lm(formula = milk ~ income + price, data = filter(milk_demand, city == 1)) 
% (Rural):  lm(formula = milk ~ income + price, data = filter(milk_demand, city == 0)) 
%
%%%%%%%%%%%%%%%%%%%%%%%%%%%%%%%%%%%%%%%%%%%%%%%%%%%%%%%%%%%%%%%%%%%%%%%%%%%%%%%%%%%%%%%
\begin{tabular}{lD{.}{.}{3}cD{.}{.}{3}cD{.}{.}{3}}
\toprule
&\multicolumn{1}{c}{(All)}&&\multicolumn{1}{c}{(Urban)}&&\multicolumn{1}{c}{(Rural)}\\
\midrule
(Intercept)&2.198&&13.570^{*}&&-5.089\\
&(4.648)&&(5.849)&&(6.726)\\
income&0.203^{***}&&0.085&&0.229^{**}\\
&(0.057)&&(0.082)&&(0.075)\\
price&-0.252&&-0.316&&-0.061\\
&(0.181)&&(0.219)&&(0.268)\\
\midrule
R-squared&0.1&&0.1&&0.2\\
adj. R-squared&0.1&&0.0&&0.1\\
sigma&5.6&&4.7&&5.8\\
F&6.4&&1.2&&4.8\\
P-value&0.0&&0.3&&0.0\\
RSS&3032.1&&1007.6&&1625.1\\
n observations&100&&48&&52\\
\bottomrule
\end{tabular}



\begin{enumerate}
\item Проверьте значимость в целом регрессии (All) на 5\%-ом уровне значимости.
\item На 5\%-ом уровне значимости проверьте гипотезу, что зависимость спроса на молоко является единой для городской и сельской местности.
\end{enumerate}

\newpage


\item У Эконометрессы Глафиры было четыре наблюдения и она решила оценить модель парной регрессии $y_i = \beta_1 + \beta_2 x_i + u_i$. Её подруга эконометресса Анжелла решила, что четыре наблюдения — мало, и поэтому учла каждое наблюдение 10 раз, так что в результате у неё вышло 40 наблюдений.

\begin{enumerate}
\item Во сколько раз будут отличаться оценки $\hb_1$ у Глафиры и Анжелы? Оценки $\hb_2$?
\item Во сколько раз у будут отличаться $RSS$? $TSS$? $ESS$? $R^2$?
\end{enumerate}


\item У эконометрессы Агнессы есть дамми-переменная $male_i$, равная 1 для мужчин, и дамми-переменная $female_i$, равная 1 для женщин. Зависимая переменная $y_i$ — доход индивида.
\[
A: \hy_i = \hat{\beta}male_i
\]
\[
B: \hy_i = \hat{\gamma}female_i
\]
\[
C: \hy_i = \hat{\alpha}_1male_i+\hat{\alpha}_2 female_i
\]
\[
D: \hy_i =  \hat{\delta}_1 + \hat{\delta}_2male_i
\]

\begin{enumerate}
\item Проинтерпретируйте оценки коэффициентов во всех регрессиях.
\item Как связаны между собой оценки коэффициентов в регрессиях C и D?
\end{enumerate}


\newpage

\item По 2040 наблюдениям оценена модель зависимости стоимости квартиры в Москве (в 1000\$) от общего метража и метража жилой площади.
% latex table generated in R 3.3.2 by xtable 1.8-2 package
% Wed Feb 15 10:22:14 2017
\begin{table}[ht]
\centering
\begin{tabular}{rrrrr}
  \hline
 & Estimate & Std. Error & t value & Pr($>$$|$t$|$) \\ 
  \hline
Константа & -88.81 & 4.37 & -20.34 & 0.00 \\ 
  Общая площадь & 1.70 & 0.10 & 17.78 & 0.00 \\ 
  Жилая площадь & 1.99 & 0.18 & 10.89 & 0.00 \\ 
   \hline
\end{tabular}
\end{table}


Сумма квадратов остатков равна $RSS=\ensuremath{2.2216891\times 10^{6}}$. Оценка ковариационной матрицы $\widehat{Var}(\hat{\beta})$ имеет вид
% latex table generated in R 3.3.2 by xtable 1.8-2 package
% Wed Feb 15 10:22:14 2017
\begin{table}[ht]
\centering
\begin{tabular}{rrrr}
  \hline
 & (Intercept) & totsp & livesp \\ 
  \hline
(Intercept) & 19.0726 & 0.0315 & -0.4498 \\ 
  totsp & 0.0315 & 0.0091 & -0.0151 \\ 
  livesp & -0.4498 & -0.0151 & 0.0335 \\ 
   \hline
\end{tabular}
\end{table}


\begin{enumerate}
\item Постройте 95\%-ый доверительный интервал для ожидаемой стоимости квартиры с жилой площадью $30$ м$^2$ и общей площадью $60$ м$^2$.
\item Постройте 95\%-ый прогнозный интервал для фактической стоимости квартиры с жилой площадью $30$ м$^2$ и общей площадью $60$ м$^2$.
\end{enumerate}

\end{enumerate}

\end{document}
