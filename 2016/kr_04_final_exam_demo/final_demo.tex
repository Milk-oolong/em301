\documentclass[12pt]{article}\usepackage[]{graphicx}\usepackage[svgnames]{xcolor}
%% maxwidth is the original width if it is less than linewidth
%% otherwise use linewidth (to make sure the graphics do not exceed the margin)
\makeatletter
\def\maxwidth{ %
  \ifdim\Gin@nat@width>\linewidth
    \linewidth
  \else
    \Gin@nat@width
  \fi
}
\makeatother

\definecolor{fgcolor}{rgb}{0.345, 0.345, 0.345}
\newcommand{\hlnum}[1]{\textcolor[rgb]{0.686,0.059,0.569}{#1}}%
\newcommand{\hlstr}[1]{\textcolor[rgb]{0.192,0.494,0.8}{#1}}%
\newcommand{\hlcom}[1]{\textcolor[rgb]{0.678,0.584,0.686}{\textit{#1}}}%
\newcommand{\hlopt}[1]{\textcolor[rgb]{0,0,0}{#1}}%
\newcommand{\hlstd}[1]{\textcolor[rgb]{0.345,0.345,0.345}{#1}}%
\newcommand{\hlkwa}[1]{\textcolor[rgb]{0.161,0.373,0.58}{\textbf{#1}}}%
\newcommand{\hlkwb}[1]{\textcolor[rgb]{0.69,0.353,0.396}{#1}}%
\newcommand{\hlkwc}[1]{\textcolor[rgb]{0.333,0.667,0.333}{#1}}%
\newcommand{\hlkwd}[1]{\textcolor[rgb]{0.737,0.353,0.396}{\textbf{#1}}}%
\let\hlipl\hlkwb

\usepackage{framed}
\makeatletter
\newenvironment{kframe}{%
 \def\at@end@of@kframe{}%
 \ifinner\ifhmode%
  \def\at@end@of@kframe{\end{minipage}}%
  \begin{minipage}{\columnwidth}%
 \fi\fi%
 \def\FrameCommand##1{\hskip\@totalleftmargin \hskip-\fboxsep
 \colorbox{shadecolor}{##1}\hskip-\fboxsep
     % There is no \\@totalrightmargin, so:
     \hskip-\linewidth \hskip-\@totalleftmargin \hskip\columnwidth}%
 \MakeFramed {\advance\hsize-\width
   \@totalleftmargin\z@ \linewidth\hsize
   \@setminipage}}%
 {\par\unskip\endMakeFramed%
 \at@end@of@kframe}
\makeatother

\definecolor{shadecolor}{rgb}{.97, .97, .97}
\definecolor{messagecolor}{rgb}{0, 0, 0}
\definecolor{warningcolor}{rgb}{1, 0, 1}
\definecolor{errorcolor}{rgb}{1, 0, 0}
\newenvironment{knitrout}{}{} % an empty environment to be redefined in TeX

\usepackage{alltt}



\usepackage[top=3cm, left=2cm, right=2cm]{geometry} % размер текста на странице

\usepackage[box, % запрет на перенос вопросов
%nopage,
insidebox, % ставим буквы в квадратики
separateanswersheet, % добавляем бланк ответов
nowatermark, % отсутствие надписи "Черновик"
%indivanswers,  % показываем верные ответы
%answers,
lang=RU,
nopage, % убираем оформление страницы (идентификаторы для распознавания)
completemulti]{automultiplechoice}

\usepackage{tikz} % картинки в tikz
\usepackage{microtype} % свешивание пунктуации

\usepackage{dcolumn} % для разделения по десятичной точке (для функции mtable)
\usepackage{comment} % для многострочных комментариев

\usepackage{array} % для столбцов фиксированной ширины

\usepackage{indentfirst} % отступ в первом параграфе

\usepackage{sectsty} % для центрирования названий частей
\allsectionsfont{\centering}

\usepackage{amsmath, amsfonts} % куча стандартных математических плюшек


\usepackage{multicol} % текст в несколько колонок

\usepackage{lastpage} % чтобы узнать номер последней страницы

\usepackage{enumitem} % дополнительные плюшки для списков
%  например \begin{enumerate}[resume] позволяет продолжить нумерацию в новом списке










\usepackage{fancyhdr} % весёлые колонтитулы
\pagestyle{fancy}
\lhead{Эконометрика, финальный экзамен}
\chead{}
\rhead{} % URFU version
%\rhead{19.06.2017, демо-вариант} % var_no + sample(c(0, 10), 1)
\lfoot{}
\cfoot{}
\rfoot{\thepage/\pageref{LastPage}}
\renewcommand{\headrulewidth}{0.4pt}
\renewcommand{\footrulewidth}{0.4pt}



\usepackage{todonotes} % для вставки в документ заметок о том, что осталось сделать
% \todo{Здесь надо коэффициенты исправить}
% \missingfigure{Здесь будет Последний день Помпеи}
% \listoftodos — печатает все поставленные \todo'шки


% более красивые таблицы
\usepackage{booktabs}
% заповеди из докупентации:
% 1. Не используйте вертикальные линни
% 2. Не используйте двойные линии
% 3. Единицы измерения - в шапку таблицы
% 4. Не сокращайте .1 вместо 0.1
% 5. Повторяющееся значение повторяйте, а не говорите "то же"



\usepackage{fontspec}
\usepackage{polyglossia}

\setmainlanguage{russian}
\setotherlanguages{english}

% download "Linux Libertine" fonts:
% http://www.linuxlibertine.org/index.php?id=91&L=1
\setmainfont{Linux Libertine O} % or Helvetica, Arial, Cambria
% why do we need \newfontfamily:
% http://tex.stackexchange.com/questions/91507/
\newfontfamily{\cyrillicfonttt}{Linux Libertine O}



%% эконометрические сокращения
\DeclareMathOperator{\plim}{plim}
\DeclareMathOperator{\Cov}{Cov}
\DeclareMathOperator{\Corr}{Corr}
\DeclareMathOperator{\Var}{Var}
\DeclareMathOperator{\E}{E}
\def \hb{\hat{\beta}}
\def \hs{\hat{\sigma}}
\def \htheta{\hat{\theta}}
\def \s{\sigma}
\def \hy{\hat{y}}
\def \hY{\hat{Y}}
\def \v1{\vec{1}}
\def \e{\varepsilon}
\def \he{\hat{\e}}
\def \z{z}

\def \sVar{\widehat{\Var}}
\def \sCorr{\widehat{\Corr}}
\def \sCov{\widehat{\Cov}}



\def \hVar{\widehat{\Var}}
\def \hCorr{\widehat{\Corr}}
\def \hCov{\widehat{\Cov}}
\def \cN{\mathcal{N}}


\AddEnumerateCounter{\asbuk}{\russian@alph}{щ} % для списков с русскими буквами
\setlist[enumerate, 2]{label=\asbuk*),ref=\asbuk*}
\IfFileExists{upquote.sty}{\usepackage{upquote}}{}
\begin{document}








\element{final_spring_2017_demo}{ % в фигурных скобках название группы вопросов
%  %\AMCnoCompleteMulti
\begin{questionmult}{1} % тип вопроса (questionmult — множественный выбор) и в фигурных — номер вопроса
Если основная гипотеза в тесте Дики-Фуллера отвергается, то временной ряд является
\begin{multicols}{2} % располагаем ответы в 3 колонки
\begin{choices} % опция [o] не рандомизирует порядок ответов
       \correctchoice{стационарным}
       \wrongchoice{нестационарным}
       \wrongchoice{коинтегрированным}
       \wrongchoice{стационарным в первых разностях}
       \wrongchoice{нормально распределённым}
    \end{choices}
   \end{multicols}
\end{questionmult}
}



\element{final_spring_2017_demo}{ % в фигурных скобках название группы вопросов
%  %\AMCnoCompleteMulti
\begin{questionmult}{2} % тип вопроса (questionmult — множественный выбор) и в фигурных — номер вопроса
Взятием разностей может быть сведен к стационарному

%\begin{multicols}{3} % располагаем ответы в 3 колонки
\begin{choices} % опция [o] не рандомизирует порядок ответов
       \correctchoice{как временной ряд с детерминированным трендом, так и со случайным трендом}
       \wrongchoice{только временной ряд с детерминированным трендом}
       \wrongchoice{только временной ряд со случайным трендом}
       \wrongchoice{ни временной ряд с детерминированным трендом, ни со случайным трендом}
       \wrongchoice{только коинтегрированный ряд}
    \end{choices}
%   \end{multicols}
\end{questionmult}
}

\element{final_spring_2017_demo}{ % в фигурных скобках название группы вопросов
%  %\AMCnoCompleteMulti
\begin{questionmult}{3} % тип вопроса (questionmult — множественный выбор) и в фигурных — номер вопроса
Если в регрессии обнаружена автокорреляция типа AR(1), то статистика Дарбина-Уотсона и оценка коэффициента автокорреляции $\hat\rho$ связаны между собой соотношением

\begin{multicols}{3} % располагаем ответы в 3 колонки
\begin{choices} % опция [o] не рандомизирует порядок ответов
       \correctchoice{$DW \approx 2(1-\hat\rho)$}
       \wrongchoice{$DW \approx \hat\rho / 2$}
       \wrongchoice{$DW \approx \hat\rho$}
       \wrongchoice{$\hat\rho \approx DW/2$}
       \wrongchoice{$\hat\rho \approx 2(1-DW)$}
    \end{choices}
\end{multicols}
\end{questionmult}
}


\element{final_spring_2017_demo}{ % в фигурных скобках название группы вопросов
%  %\AMCnoCompleteMulti
\begin{questionmult}{4} % тип вопроса (questionmult — множественный выбор) и в фигурных — номер вопроса
Выберите верное утверждение о модели бинарного выбора:
%\begin{multicols}{3} % располагаем ответы в 3 колонки
\begin{choices} % опция [o] не рандомизирует порядок ответов
       \correctchoice{недостатком линейной вероятностной модели является возможная нереалистичность значений вероятности}
       \wrongchoice{нельзя включать в качестве независимых дамми-переменные}
       \wrongchoice{значимость коэффициентов проверяется с помощью статистики, имеющей $t$-распределение}
       \wrongchoice{ROC кривая является выпуклой для любой логит-модели}
       \wrongchoice{оценки коэффициентов логит и пробит моделей всегда имеют один и тот же знак}
    \end{choices}
 %  \end{multicols}
\end{questionmult}
}


\element{final_spring_2017_demo}{ % в фигурных скобках название группы вопросов
%  %\AMCnoCompleteMulti
\begin{questionmult}{5} % тип вопроса (questionmult — множественный выбор) и в фигурных — номер вопроса
При оценивании модели $Y_t = X_t' \beta + u_t$ была обнаружена автокорреляция первого порядка с $\hat\rho = -0.6$. Чтобы провести корректное оценивание, можно применить метод наименьших квадратов  к преобразованным данным. При этом первое наблюдение окажется домноженным на 
\begin{multicols}{3} % располагаем ответы в 3 колонки
\begin{choices} % опция [o] не рандомизирует порядок ответов
        \correctchoice{$0.8$}
       \wrongchoice{$0.6$}
       \wrongchoice{$-0.6$}
       \wrongchoice{$0.4$}
       \wrongchoice{$\sqrt{0.6}$}
        \wrongchoice{$\sqrt{0.84}$}
    \end{choices}
\end{multicols}
\end{questionmult}
}



\element{final_spring_2017_demo}{ % в фигурных скобках название группы вопросов
%  %\AMCnoCompleteMulti
\begin{questionmult}{6} % тип вопроса (questionmult — множественный выбор) и в фигурных — номер вопроса
Условие порядка для любого уравнения из системы может быть сформулировано следующим образом.
Число эндогенных переменных, включенных в уравнение, уменьшенное на 1, должно быть
%\begin{multicols}{3} % располагаем ответы в 3 колонки
\begin{choices} % опция [o] не рандомизирует порядок ответов
       \correctchoice{не больше числа экзогенных переменных, исключенных из этого уравнения}
       \wrongchoice{не больше числа экзогенных переменных, включенных в это уравнение}
       \wrongchoice{не меньше числа экзогенных переменных, включенных в это уравнение}
       \wrongchoice{не меньше числа экзогенных переменных, исключенных из этого уравнения}
       \wrongchoice{не больше числа эндогенных переменных, исключенных из этого уравнения}
       \wrongchoice{не больше числа эндогенных переменных, включенных в это уравнение}
    \end{choices}
   %\end{multicols}
\end{questionmult}
}


\element{final_spring_2017_demo}{ % в фигурных скобках название группы вопросов
%  %\AMCnoCompleteMulti
\begin{questionmult}{7} % тип вопроса (questionmult — множественный выбор) и в фигурных — номер вопроса
Инструмент $Z_t$ для оценивания динамической модели $Y_t = \beta_1 + \beta_2 X_t + \beta_3 Y_{t-1} + u_t$ с экзогенным вектором $X$ и AR(1) процессом в ошибках $u_t$ должен удовлетворять требованию
\begin{multicols}{3} % располагаем ответы в 3 колонки
\begin{choices} % опция [o] не рандомизирует порядок ответов
       \correctchoice{$\Corr(Y_{t-1}, Z_t) \to 1$}
       \wrongchoice{$\Corr(Y_{t-1}, Z_t)=0$}
       \wrongchoice{$\Corr(X_t, Z_t)=0$}
       \wrongchoice{$\Corr(u_t, Z_t)=0$}
       \wrongchoice{$\Corr(u_t, Z_t) \to 1$}
    \end{choices}
   \end{multicols}
\end{questionmult}
}

\element{final_spring_2017_demo}{ % в фигурных скобках название группы вопросов
%  %\AMCnoCompleteMulti
\begin{questionmult}{8} % тип вопроса (questionmult — множественный выбор) и в фигурных — номер вопроса
Рассмотрим модель $Y_i = {\beta_1} + {\beta_2}{X_{i2}} + {\beta_3}{X_{i3}} + {\beta_4}{X_{i4}} + {\beta_5}{X_{i5}} + u_i$.
Гипотезу 
\[
\begin{cases}
{\beta_2} + {\beta _3} = 1\\
{\beta_5} = 0 \\
\end{cases}
\]
можно проверить с помощью оценки дополнительной модели 


\begin{multicols}{2} % располагаем ответы в 3 колонки
\begin{choices} % опция [o] не рандомизирует порядок ответов
       \correctchoice{$Y_i - X_{i3} = \beta_1 + \beta_2 (X_{i2} - X_{i3}) + \beta_4 X_{i4} + u_i$}
       \wrongchoice{$Y_i - X_{i2} = \beta_1 + \beta_2 (X_{i2} - X_{i3}) + \beta_4 X_{i4} + u_i$}
       \wrongchoice{$Y_i - X_{i3} = \beta_1 + \beta_2 (X_{i2} + X_{i3}) + \beta_4 X_{i4} + u_i$}
       \wrongchoice{$Y_i - \beta_2 = \beta_1 + \beta_2 (X_{i2} + X_{i3}) + \beta_4 X_{i4} + u_i$}
       \wrongchoice{$Y_i = \beta_1 + \beta_2 (X_{i2} + X_{i3} - 1) + \beta_4 X_{i4} + u_i$}
    \end{choices}
   \end{multicols}
\end{questionmult}
}

\element{final_spring_2017_demo}{ % в фигурных скобках название группы вопросов
%  %\AMCnoCompleteMulti
\begin{questionmult}{9} % тип вопроса (questionmult — множественный выбор) и в фигурных — номер вопроса
К несостоятельности МНК-оценок вектора коэффициентов приводит
\begin{multicols}{2} % располагаем ответы в 3 колонки
\begin{choices} % опция [o] не рандомизирует порядок ответов
       \correctchoice{эндогенность одного из регрессоров}
       \wrongchoice{корреляция ошибок по схеме AR(1)}
       \wrongchoice{корреляция ошибок по схеме MA(1)}
       \wrongchoice{нестрогая мультиколлинеарность}
       \wrongchoice{корреляция между регрессорами}
       \wrongchoice{условная гетероскедастичность ошибок}
    \end{choices}
   \end{multicols}
\end{questionmult}
}


\element{final_spring_2017_demo}{ % в фигурных скобках название группы вопросов
%  %\AMCnoCompleteMulti
\begin{questionmult}{10} % тип вопроса (questionmult — множественный выбор) и в фигурных — номер вопроса
Процесс $u_t$ является белым шумом. Нестационарным является процесс
\begin{multicols}{2} % располагаем ответы в 3 колонки
\begin{choices} % опция [o] не рандомизирует порядок ответов
       \correctchoice{$Y_t = - Y_{t-1} + u_t$}
       \wrongchoice{$Y_t$ независимо и одинаково распределены $\cN(7; 16)$}
       \wrongchoice{$Y_t = u_t + 2u_{t-1}$}
       \wrongchoice{$Y_t = 7 + u_t + 0.2 u_{t-1} - 1.2 u_t{-2}$}
       \wrongchoice{$Y_t = 5 + 0.1 Y_{t-1} + u_{t} + 0.2 u_{t-1}$}
    \end{choices}
   \end{multicols}
\end{questionmult}
}


\section*{Часть 1. Тест.}

\onecopy{1}{

\cleargroup{combat}
\copygroup[10]{final_spring_2017_demo}{combat}
%\shufflegroup{demo_a}
\insertgroup{combat}

}

\section*{Часть 2. Задачи.}




\begin{enumerate}

\item Величины $X_i$ равномерны на отрезке $[-a; 3a]$ и независимы. Есть несколько наблюдений, $X_1=0.5$, $X_2=0.7$, $X_3=-0.1$.

\begin{enumerate}
\item Найдите $\E(X_i)$ и $\E(|X_i|)$
\item Постройте оценку параметра $a$ методом моментов, используя $\E(|X_i|)$
\item Постройте оценку параметра $a$ обобщённым методом моментов, используя моменты $\E(X_i)$, $\E(|X_i|)$ и взвешивающую матрицу
\[
W=\begin{pmatrix}
2 & 0 \\
0 & 1 \\
\end{pmatrix}.
\]
\end{enumerate}


\item Рассмотрим логит-модель, задаваемую системой
\[
\begin{cases}
Y_i = 
\begin{cases}
1, \text{ если } Y_i^* \geq 0; \\
0, \text{ иначе;} \\
\end{cases} \\
Y_i^* = \beta_1 + \beta_2 X_i + u_i \\
\end{cases}.
\]



\begin{enumerate}
\item Выпишите функцию правдоподобия для набора из четырёх наблюдений: $(X_1, Y_1) = (4, 1)$, $(X_2, Y_2) = (0, 0)$,  $(X_3, Y_3) = (2, 1)$,  $(X_4, Y_4) = (3, 0)$.
\item Оценки коэффициентов равны $\hb_1 = -1.95$ и $\hb_2 = 0.85$. Оцените вероятность того, что $Y_5 = 1$ при $X_5 = 1$.
\end{enumerate}


\item Фирмы определяют необходимый запас товаров $Y_i$ в зависимости от ожидаемых годовых продаж $X^e_i$, используя линейную форму зависимости $Y_i = \beta_1  + \beta_2 X^e_i + \e_i$.


Исследователю доступны только данные о реальных продажах $X_i = X^e_i + u_i$, где ошибки $u_i$ распределены независимо от $X_i$ и удовлетворяют условию теоремы Гаусса–Маркова.  

\begin{enumerate}
\item Какие проблемы возникнут при оценке исходной модели с помощью МНК, если вместо данных по $X^e_i$ будут использованы данные по $X_i$?
\item Каков возможный способ решения этих проблем?
\end{enumerate}







\item Рассмотрим стационарный случайный процесс $y_t$, удовлетворяющий уравнению
\[
y_t = 3 + 0.7 y_{t-1} - 0.1 y_{t-2} + u_t, 
\]
где $u_t$ — белый шум с дисперсией $5$.

Найдите $\E(y_t)$, $\Var(y_t)$, $\Cov(y_t, y_{t-1})$, $\Cov(y_t, y_{t-2})$.

\item Что такое коинтегрированные временные ряды? Как проверить, являются ли два временных ряда коинтегрированными?

\item Модели панельных данных с фиксированными эффектами: определение, способы оценивания.

\end{enumerate}

\end{document}
