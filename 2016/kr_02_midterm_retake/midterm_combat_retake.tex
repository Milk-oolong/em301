\documentclass[12pt]{article}\usepackage[]{graphicx}\usepackage[svgnames]{xcolor}
%% maxwidth is the original width if it is less than linewidth
%% otherwise use linewidth (to make sure the graphics do not exceed the margin)
\makeatletter
\def\maxwidth{ %
  \ifdim\Gin@nat@width>\linewidth
    \linewidth
  \else
    \Gin@nat@width
  \fi
}
\makeatother

\definecolor{fgcolor}{rgb}{0.345, 0.345, 0.345}
\newcommand{\hlnum}[1]{\textcolor[rgb]{0.686,0.059,0.569}{#1}}%
\newcommand{\hlstr}[1]{\textcolor[rgb]{0.192,0.494,0.8}{#1}}%
\newcommand{\hlcom}[1]{\textcolor[rgb]{0.678,0.584,0.686}{\textit{#1}}}%
\newcommand{\hlopt}[1]{\textcolor[rgb]{0,0,0}{#1}}%
\newcommand{\hlstd}[1]{\textcolor[rgb]{0.345,0.345,0.345}{#1}}%
\newcommand{\hlkwa}[1]{\textcolor[rgb]{0.161,0.373,0.58}{\textbf{#1}}}%
\newcommand{\hlkwb}[1]{\textcolor[rgb]{0.69,0.353,0.396}{#1}}%
\newcommand{\hlkwc}[1]{\textcolor[rgb]{0.333,0.667,0.333}{#1}}%
\newcommand{\hlkwd}[1]{\textcolor[rgb]{0.737,0.353,0.396}{\textbf{#1}}}%
\let\hlipl\hlkwb

\usepackage{framed}
\makeatletter
\newenvironment{kframe}{%
 \def\at@end@of@kframe{}%
 \ifinner\ifhmode%
  \def\at@end@of@kframe{\end{minipage}}%
  \begin{minipage}{\columnwidth}%
 \fi\fi%
 \def\FrameCommand##1{\hskip\@totalleftmargin \hskip-\fboxsep
 \colorbox{shadecolor}{##1}\hskip-\fboxsep
     % There is no \\@totalrightmargin, so:
     \hskip-\linewidth \hskip-\@totalleftmargin \hskip\columnwidth}%
 \MakeFramed {\advance\hsize-\width
   \@totalleftmargin\z@ \linewidth\hsize
   \@setminipage}}%
 {\par\unskip\endMakeFramed%
 \at@end@of@kframe}
\makeatother

\definecolor{shadecolor}{rgb}{.97, .97, .97}
\definecolor{messagecolor}{rgb}{0, 0, 0}
\definecolor{warningcolor}{rgb}{1, 0, 1}
\definecolor{errorcolor}{rgb}{1, 0, 0}
\newenvironment{knitrout}{}{} % an empty environment to be redefined in TeX

\usepackage{alltt}



\usepackage[top=3cm, left=2cm, right=2cm]{geometry} % размер текста на странице

\usepackage[box, % запрет на перенос вопросов
%nopage,
insidebox, % ставим буквы в квадратики
separateanswersheet, % добавляем бланк ответов
nowatermark, % отсутствие надписи "Черновик"
% indivanswers,  % показываем верные ответы
%answers,
lang=RU,
nopage, % убираем оформление страницы (идентификаторы для распознавания)
completemulti]{automultiplechoice}

\usepackage{tikz} % картинки в tikz
\usepackage{microtype} % свешивание пунктуации

\usepackage{dcolumn} % для разделения по десятичной точке (для функции mtable)
\usepackage{comment} % для многострочных комментариев

\usepackage{array} % для столбцов фиксированной ширины

\usepackage{indentfirst} % отступ в первом параграфе

\usepackage{sectsty} % для центрирования названий частей
\allsectionsfont{\centering}

\usepackage{amsmath, amsfonts} % куча стандартных математических плюшек


\usepackage{multicol} % текст в несколько колонок

\usepackage{lastpage} % чтобы узнать номер последней страницы

\usepackage{enumitem} % дополнительные плюшки для списков
%  например \begin{enumerate}[resume] позволяет продолжить нумерацию в новом списке










\usepackage{fancyhdr} % весёлые колонтитулы
\pagestyle{fancy}
\lhead{Эконометрика, пересдача}
\chead{}
\rhead{16.01.2016}
\lfoot{}
\cfoot{}
\rfoot{\thepage/\pageref{LastPage}}
\renewcommand{\headrulewidth}{0.4pt}
\renewcommand{\footrulewidth}{0.4pt}



\usepackage{todonotes} % для вставки в документ заметок о том, что осталось сделать
% \todo{Здесь надо коэффициенты исправить}
% \missingfigure{Здесь будет Последний день Помпеи}
% \listoftodos — печатает все поставленные \todo'шки


% более красивые таблицы
\usepackage{booktabs}
% заповеди из докупентации:
% 1. Не используйте вертикальные линни
% 2. Не используйте двойные линии
% 3. Единицы измерения - в шапку таблицы
% 4. Не сокращайте .1 вместо 0.1
% 5. Повторяющееся значение повторяйте, а не говорите "то же"



\usepackage{fontspec}
\usepackage{polyglossia}

\setmainlanguage{russian}
\setotherlanguages{english}

% download "Linux Libertine" fonts:
% http://www.linuxlibertine.org/index.php?id=91&L=1
\setmainfont{Linux Libertine O} % or Helvetica, Arial, Cambria
% why do we need \newfontfamily:
% http://tex.stackexchange.com/questions/91507/
\newfontfamily{\cyrillicfonttt}{Linux Libertine O}

\AddEnumerateCounter{\asbuk}{\russian@alph}{щ} % для списков с русскими буквами


%% эконометрические сокращения
\DeclareMathOperator{\plim}{plim}
\DeclareMathOperator{\Cov}{Cov}
\DeclareMathOperator{\Corr}{Corr}
\DeclareMathOperator{\Var}{Var}
\DeclareMathOperator{\E}{E}
\def \hb{\hat{\beta}}
\def \hs{\hat{\sigma}}
\def \htheta{\hat{\theta}}
\def \s{\sigma}
\def \hy{\hat{y}}
\def \hY{\hat{Y}}
\def \v1{\vec{1}}
\def \e{\varepsilon}
\def \he{\hat{\e}}
\def \z{z}

\def \sVar{\widehat{\Var}}
\def \sCorr{\widehat{\Corr}}
\def \sCov{\widehat{\Cov}}



\def \hVar{\widehat{\Var}}
\def \hCorr{\widehat{\Corr}}
\def \hCov{\widehat{\Cov}}
\def \cN{\mathcal{N}}


\AddEnumerateCounter{\asbuk}{\russian@alph}{щ} % для списков с русскими буквами
\setlist[enumerate, 2]{label=\asbuk*),ref=\asbuk*}
\IfFileExists{upquote.sty}{\usepackage{upquote}}{}
\begin{document}








\element{midterm_16}{ % в фигурных скобках название группы вопросов
%  %\AMCnoCompleteMulti
\begin{questionmult}{1} % тип вопроса (questionmult — множественный выбор) и в фигурных — номер вопроса
В регрессии с константой и тремя объясняющими переменными сумма квадратов остатков равна $310$, а число наблюдений равно $35$. Точечная оценка дисперсии случайной составляющей равна
\begin{multicols}{3} % располагаем ответы в 3 колонки
\begin{choices} % опция [o] не рандомизирует порядок ответов
       \correctchoice{$10$}
       \wrongchoice{$11$}
       \wrongchoice{$12$}
       \wrongchoice{$\sqrt{10}$}
       \wrongchoice{$\sqrt{11}$}
       \wrongchoice{$\sqrt{12}$}
    \end{choices}
   \end{multicols}
\end{questionmult}
}


\element{midterm_16}{ % в фигурных скобках название группы вопросов
%  %\AMCnoCompleteMulti
\begin{questionmult}{2} % тип вопроса (questionmult — множественный выбор) и в фигурных — номер вопроса
Если для регрессора используется преобразование Бокса-Кокса с параметром $\theta=1$, а для зависимой переменной — с параметром $\lambda = 1$, то регрессионное уравнение представимо в виде
\begin{multicols}{3} % располагаем ответы в 3 колонки
\begin{choices} % опция [o] не рандомизирует порядок ответов
       \wrongchoice{$Y_i = \beta_1 + \beta_2 \frac{1}{X_i} + u_i$}
       \wrongchoice{$\ln Y_i = \beta_1 + \beta_2 \ln X_i + u_i$}
       \wrongchoice{$\ln Y_i = \beta_1 + \beta_2 X_i + u_i$}
       \wrongchoice{$Y_i = \beta_1 + \beta_2 \ln X_i + u_i$}
       \correctchoice{$Y_i = \beta_1 + \beta_2 X_i + u_i$}
       \wrongchoice{$\ln Y_i = \beta_1 - \beta_2 \ln X_i + u_i$}
    \end{choices}
   \end{multicols}
\end{questionmult}
}

\element{midterm_16}{ % в фигурных скобках название группы вопросов
%  %\AMCnoCompleteMulti
\begin{questionmult}{3} % тип вопроса (questionmult — множественный выбор) и в фигурных — номер вопроса
Известно, что регрессоры $X$ и $Z$ ортогональны, а истинная зависимость описывается уравнением $Y_i = \alpha_1 + \alpha_2 X_i + \alpha_3 Z_i + u_i$. Исследователь оценивает с помощью МНК две регрессии: $\hat Y_i = \hb_1 + \hb_2 X_i$ и $\hat Y_i = \hat \gamma_1 + \hat \gamma_2 Z_i$. При этом
%\begin{multicols}{3} % располагаем ответы в 3 колонки
\begin{choices} % опция [o] не рандомизирует порядок ответов
       \wrongchoice{$\hb_2$ — несмещённая оценка для $\alpha_3$; $\hat \gamma_2$ — несмещённая оценка для $\alpha_2$}
       \wrongchoice{$\hb_2$ — смещённая оценка для $\alpha_2$; $\hat \gamma_2$ — смещённая оценка для $\alpha_3$}
       \wrongchoice{$\hb_2$ — несмещённая оценка для $\alpha_2$; $\hat \gamma_2$ — смещённая оценка для $\alpha_3$}
       \wrongchoice{$\hb_2$ — смещённая оценка для $\alpha_2$; $\hat \gamma_2$ — несмещённая оценка для $\alpha_3$}
       \wrongchoice{$\hb_2$ — эффективная оценка для $\alpha_2$; $\hat \gamma_2$ — эффективная оценка для $\alpha_3$}
    \end{choices}
   %\end{multicols}
\end{questionmult}
}


\element{midterm_16}{ % в фигурных скобках название группы вопросов
%  %\AMCnoCompleteMulti
\begin{questionmult}{4} % тип вопроса (questionmult — множественный выбор) и в фигурных — номер вопроса
Гипотеза о том, что одновременно $\beta_1 + \beta_2 =1$  и $\beta_3=0$ в  линейной регрессии c $7$-ю оцениваемыми коэффициентами, построенной по $n$ наблюдениям, проверяется с помощью  статистики, имеющей распределение
\begin{multicols}{3} % располагаем ответы в 3 колонки
\begin{choices} % опция [o] не рандомизирует порядок ответов
       \correctchoice{$F_{2, n-7}$}
       \wrongchoice{$F_{1, n}$}
       \wrongchoice{$F_{1, n-7}$}
       \wrongchoice{$F_{1, n-6}$}
       \wrongchoice{$F_{1, n-2}$}
       \wrongchoice{Петрова-Водкина}
    \end{choices}
   \end{multicols}
\end{questionmult}
}

\element{midterm_16}{ % в фигурных скобках название группы вопросов
%  %\AMCnoCompleteMulti
\begin{questionmult}{5} % тип вопроса (questionmult — множественный выбор) и в фигурных — номер вопроса
Элеонора исследует зависимость цены номера в отеле от звёздности отеля, $star$, (от 1 до 3 звёзд) и расстояния до моря, $dist$. Элеонора хочет оценить модель вида $price_i = \beta_1 + \beta_2 star_i + \beta_3 dist_i + u_i$. Чтобы считаться богиней эконометрики Элеоноре стоит
\begin{multicols}{1} % располагаем ответы в 3 колонки
\begin{choices} % опция [o] не рандомизирует порядок ответов
        \wrongchoice{заменить переменную $star_i$ на дамми-переменные $one_i$ и $two_i$, равные 1 для отелей с одной и двумя звёздами соответственно, и удалить константу}
       \wrongchoice{использовать МНК для оценки данной модели}
       \wrongchoice{добавить в модель переменную $z_i = star_i \cdot dist_i$}
       \wrongchoice{добавить в модель переменную $z_i = star^2_i$, так как эффект звёздности наверняка нелинейный}
       \wrongchoice{заменить переменную $star_i$ на дамми-переменные $one_i$, $two_i$ и $three_i$, равные 1 для отелей с одной, двумя и тремя звёздами соответственно}
        \correctchoice{добавить дамми-переменные $one_i$, $two_i$ и $three_i$, равные 1 для отелей с одной, двумя и тремя звёздами соответственно, и удалить константу}
    \end{choices}
   \end{multicols}
\end{questionmult}
}


\element{midterm_16}{ % в фигурных скобках название группы вопросов
%  %\AMCnoCompleteMulti
\begin{questionmult}{6} % тип вопроса (questionmult — множественный выбор) и в фигурных — номер вопроса
Показатель $R^2_{adj}$ можно вычислить по формуле
\begin{multicols}{3} % располагаем ответы в 3 колонки
\begin{choices} % опция [o] не рандомизирует порядок ответов
       \wrongchoice{$R^2_{adj} = (-1)\cdot \frac{k-1}{n-k} + R^2 \cdot \frac{n-2}{n-k}$}
       \wrongchoice{$R^2_{adj} = \frac{k-1}{n-k} + R^2 \cdot \frac{n-1}{n-k}$}
       \wrongchoice{$R^2_{adj} = \frac{k-1}{n-k} - R^2 \cdot \frac{n-1}{n-k}$}
       \wrongchoice{$R^2_{adj} = \frac{k-1}{n-k} + R^2 \cdot \frac{n-k}{n-1}$}
       \wrongchoice{$R^2_{adj} = \frac{k}{n-k} + R^2 \cdot \frac{n-1}{n-k}$}
       \wrongchoice{$R^2_{adj} = \frac{n-k}{k-1} + R^2 \cdot \frac{n-1}{n-k}$}
    \end{choices}
   \end{multicols}
\end{questionmult}
}

\element{midterm_16}{ % в фигурных скобках название группы вопросов
%  %\AMCnoCompleteMulti
\begin{questionmult}{7} % тип вопроса (questionmult — множественный выбор) и в фигурных — номер вопроса
Если гипотеза $\beta_2 + 3\beta_3 = 1$ верна, то модель $\ln Y_i = \beta_1 + \beta_2 \ln X_i + \beta_3 \ln Z_i + u_i$ совпадает с моделью
\begin{multicols}{2} % располагаем ответы в 3 колонки
\begin{choices} % опция [o] не рандомизирует порядок ответов
       \wrongchoice{$\ln (Y_i/Z_i) = \beta_1 + \beta_2 \ln (X_i/Z_i) + u_i $}
       \wrongchoice{$\ln Y_i = \beta_1 + \beta_2 \ln (X_i/Z_i) + u_i $}
       \wrongchoice{$\ln Y_i = \beta_1 + \beta_2 \ln (Z_i/Y_i) + u_i $}
       \wrongchoice{$\ln (Y_i/Z_i) = \beta_1 + \beta_2 \ln (Y_i/Z_i) + u_i $}
       \wrongchoice{$\ln (Y_i/Z_i) = \beta_1 + \beta_2 \ln (Y_i/X_i) + u_i $}
    \end{choices}
   \end{multicols}
\end{questionmult}
}


\element{midterm_16}{ % в фигурных скобках название группы вопросов
%  %\AMCnoCompleteMulti
\begin{questionmult}{8} % тип вопроса (questionmult — множественный выбор) и в фигурных — номер вопроса
Гипотеза о неадекватности множественной регрессии проверяется с помощью статистики равной
\begin{multicols}{3} % располагаем ответы в 3 колонки
\begin{choices} % опция [o] не рандомизирует порядок ответов
       \wrongchoice{$\frac{ESS/(k-2)}{RSS/(n-k)}$}
       \wrongchoice{$\frac{\hb - \beta}{se(\hb)}$}
       \wrongchoice{$\frac{TSS/(n-1)}{RSS/(n-k)}$}
       \wrongchoice{$\frac{TSS/(n-1)}{ESS/(k-1)}$}
       \wrongchoice{$\frac{ESS}{TSS}$}
       \wrongchoice{$\frac{RSS}{TSS}$}
    \end{choices}
   \end{multicols}
\end{questionmult}
}


\element{midterm_16}{ % в фигурных скобках название группы вопросов
%  %\AMCnoCompleteMulti
\begin{questionmult}{9} % тип вопроса (questionmult — множественный выбор) и в фигурных — номер вопроса
Исследователь выполнил второй шаг в PE-тесте МакКиннона. В регрессии $\ln Y_i$ на исходные регрессоры и $Z_i = \hat Y_i - \exp(\widehat{\ln Y_i})$ коэффициент при $Z_i$ оказался значимым. А в регрессии $Y_i$ на исходные регрессоры и $W_i = \ln \hat Y_i - \widehat{\ln Y_i}$ коэффициент при $W_i$ оказался незначимым. Из результатов следует сделать вывод, что
\begin{multicols}{3} % располагаем ответы в 3 колонки
\begin{choices} % опция [o] не рандомизирует порядок ответов
       \correctchoice{следует предпочесть линейную модель}
       \wrongchoice{следует предпочесть полулогарифмеческую модель}
       \wrongchoice{следует предпочесть логарифмическую модель}
       \wrongchoice{тесты противоречат друг другу, ни одна из моделей не предпочитается}
       \wrongchoice{в исходной модели пропущен регрессор $Z_i$}
       \wrongchoice{в исходной модели пропущен регрессор $W_i$}
    \end{choices}
   \end{multicols}
\end{questionmult}
}


\element{midterm_16}{ % в фигурных скобках название группы вопросов
%  %\AMCnoCompleteMulti
\begin{questionmult}{10} % тип вопроса (questionmult — множественный выбор) и в фигурных — номер вопроса
Истинной является модель $Y_i = \beta_1 + \beta_2 X_i + u_i$. Глафира оценивает две регрессии: $\hat Y_i = \hb_1 + \hb_2X_i$ и  $\hat Y_i = \hat \gamma_1 + \hat \gamma_2 X_i + \hat \gamma_3 Z_i$ с помощью МНК. Для коэффициента $\beta_2$
\begin{multicols}{2} % располагаем ответы в 3 колонки
\begin{choices} % опция [o] не рандомизирует порядок ответов
       \wrongchoice{оценки $\hb_2$ и $\hat\gamma_2$ являются смещёнными}
       \wrongchoice{оценки $\hb_2$ и $\hat\gamma_2$ являются эффективными}
       \wrongchoice{оценка $\hb_2$ является несмещённой, а оценка  $\hat\gamma_2$ — смещённой}
       \wrongchoice{оценки $\hb_2$ и $\hat\gamma_2$ являются неэффективными}
       \wrongchoice{оценка $\hb_2$ является смещённой, а оценка  $\hat\gamma_2$ — несмещённой}
    \end{choices}
   \end{multicols}
\end{questionmult}
}


\section*{Часть 1. Тест.}

\onecopy{1}{

\cleargroup{combat}
\copygroup[10]{midterm_16}{combat}
\shufflegroup{combat}
\insertgroup{combat}

}

\section*{Часть 2. Задачи.}


\begin{enumerate}



\item На основании опроса 200 человек была оценена следующая модель:
\[
\ln(wage_i)=\beta_1 + \beta_2 exper_i + \beta_3 exper^2_i + \beta_4 sex_i + \e_i
\]

где:
\begin{itemize}
\item $wage_i$ — величина заработной платы в долларах
\item $exper_i$ — опыт работы в годах
\item $exper^2_i$ — опыт работы в годах
\item $sex_i$ — пол (1 — мужской, 0 — женский)
\end{itemize}

\begin{tabular}{lr} \toprule
Показатель & Значение \\
\midrule
$R^2$                        & 0.9 \\
Скорректированный $R^2$      & \textbf{B7} \\
Стандартная ошибка регрессии & \textbf{B6} \\
Количество наблюдений        & \textbf{B2} \\
\bottomrule
\end{tabular}

Результаты дисперсионного анализа:

\begin{tabular}{lrrrr} \toprule
            &  df           & сумма квадратов & F           & P-значение \\
\midrule
Регрессия   & 3            & \textbf{B9}     & \textbf{В5}  &   \\
Остаток     & \textbf{B1}  &  857.8  &              &       \\
Итого       & \textbf{B3}  & \textbf{B4}     &              &       \\
\bottomrule
\end{tabular}


\begin{table}[ht]
\centering
\begin{tabular}{rrrrr}
  \hline
 & Оценка & Ст. ошибка & t-статистика  \\
  \hline
Константа & 2.3 & 1.4465 & 1.59 \\
  $exper$ & \textbf{В8} & 0.4085 & 15.4214  \\
  $exper^2$ & -0.226 & 0.0283 & -7.9865  \\
  $sex$ & 1.315 & 0.2981 & \textbf{В10}  \\
   \hline
\end{tabular}
\end{table}



\begin{enumerate}
\item Найдите пропущенные числа \textbf{B1}--\textbf{B10}.

\item Как изменятся результаты оценки регрессии, если из регрессии удалить константу и добавить переменную $f_i$, равную 0 для мужчин и 1 — для женщин?
\end{enumerate}

Ответ округляйте до 2-х знаков после запятой. Кратко поясняйте, например, формулой, как были получены результаты.

\newpage
\item Исследовательница Глафира изучает зависимость спроса на молоко от цены молока и дохода семьи. В её распоряжении есть следующие переменные:

\begin{itemize}
\item $price$ — цена молока в рублях за литр
\item $income$ — ежемесячный доход семьи в тысячах рублей
\item $milk$ — расходы семьи на молоко за последние семь дней в рублях
\end{itemize}

В данных указано, проживает ли семья в сельской или городской местности. Поэтому Глафира оценила три регрессии: (All) — по всем данным, (Urban) — по городским семьям, (Rural) — по сельским семьям.

%%%%%%%%%%%%%%%%%%%%%%%%%%%%%%%%%%%%%%%%%%%%%%%%%%%%%%%%%%%%%%%%%%%%%%%%%%%%%%%%%%%%%%%
%
% Calls:
% (All):  lm(formula = milk ~ income + price, data = milk_demand) 
% (Urban):  lm(formula = milk ~ income + price, data = filter(milk_demand, city == 1)) 
% (Rural):  lm(formula = milk ~ income + price, data = filter(milk_demand, city == 0)) 
%
%%%%%%%%%%%%%%%%%%%%%%%%%%%%%%%%%%%%%%%%%%%%%%%%%%%%%%%%%%%%%%%%%%%%%%%%%%%%%%%%%%%%%%%
\begin{tabular}{lD{.}{.}{3}cD{.}{.}{3}cD{.}{.}{3}}
\toprule
&\multicolumn{1}{c}{(All)}&&\multicolumn{1}{c}{(Urban)}&&\multicolumn{1}{c}{(Rural)}\\
\midrule
(Intercept)&2.274&&12.918^{*}&&-7.818\\
&(4.622)&&(6.404)&&(5.848)\\
income&0.179^{***}&&0.023&&0.318^{***}\\
&(0.050)&&(0.068)&&(0.065)\\
price&-0.142&&-0.068&&-0.172\\
&(0.158)&&(0.227)&&(0.195)\\
\midrule
R-squared&0.1&&0.0&&0.3\\
adj. R-squared&0.1&&-0.0&&0.3\\
sigma&4.9&&4.7&&4.5\\
F&6.6&&0.1&&12.2\\
P-value&0.0&&0.9&&0.0\\
RSS&2367.1&&979.2&&1021.6\\
n observations&100&&47&&53\\
\bottomrule
\end{tabular}



\begin{enumerate}
\item Проверьте значимость в целом регрессии (All) на 5\%-ом уровне значимости.
\item На 5\%-ом уровне значимости проверьте гипотезу, что зависимость спроса на молоко является единой для городской и сельской местности.
\end{enumerate}

\newpage

\item Исследователь Луноликий оценил модель $y_i = \beta_1 + \beta_2 x_i + u_i$. На плоскости изображены наблюдения и линия регрессии:

\begin{center}
\begin{figure}
\begin{tikzpicture}[scale = 0.025]
% Created by tikzDevice version 0.10.1 on 2016-12-29 20:38:19
% !TEX encoding = UTF-8 Unicode
\definecolor{fillColor}{RGB}{255,255,255}
\path[use as bounding box,fill=fillColor,fill opacity=0.00] (0,0) rectangle (505.89,505.89);
\begin{scope}
\path[clip] (  0.00,  0.00) rectangle (505.89,505.89);
\definecolor{drawColor}{RGB}{255,255,255}
\definecolor{fillColor}{RGB}{255,255,255}

\path[draw=drawColor,line width= 0.6pt,line join=round,line cap=round,fill=fillColor] (  0.00,  0.00) rectangle (505.89,505.89);
\end{scope}
\begin{scope}
\path[clip] ( 39.04, 29.52) rectangle (500.39,500.39);
\definecolor{fillColor}{RGB}{255,255,255}

\path[fill=fillColor] ( 39.04, 29.52) rectangle (500.39,500.39);
\definecolor{drawColor}{gray}{0.92}

\path[draw=drawColor,line width= 0.3pt,line join=round] ( 39.04, 77.85) --
	(500.39, 77.85);

\path[draw=drawColor,line width= 0.3pt,line join=round] ( 39.04,163.20) --
	(500.39,163.20);

\path[draw=drawColor,line width= 0.3pt,line join=round] ( 39.04,248.54) --
	(500.39,248.54);

\path[draw=drawColor,line width= 0.3pt,line join=round] ( 39.04,333.89) --
	(500.39,333.89);

\path[draw=drawColor,line width= 0.3pt,line join=round] ( 39.04,419.24) --
	(500.39,419.24);

\path[draw=drawColor,line width= 0.3pt,line join=round] (109.35, 29.52) --
	(109.35,500.39);

\path[draw=drawColor,line width= 0.3pt,line join=round] (134.02, 29.52) --
	(134.02,500.39);

\path[draw=drawColor,line width= 0.3pt,line join=round] (183.36, 29.52) --
	(183.36,500.39);

\path[draw=drawColor,line width= 0.3pt,line join=round] (238.87, 29.52) --
	(238.87,500.39);

\path[draw=drawColor,line width= 0.3pt,line join=round] (294.38, 29.52) --
	(294.38,500.39);

\path[draw=drawColor,line width= 0.3pt,line join=round] (319.06, 29.52) --
	(319.06,500.39);

\path[draw=drawColor,line width= 0.6pt,line join=round] ( 39.04, 35.17) --
	(500.39, 35.17);

\path[draw=drawColor,line width= 0.6pt,line join=round] ( 39.04,120.52) --
	(500.39,120.52);

\path[draw=drawColor,line width= 0.6pt,line join=round] ( 39.04,205.87) --
	(500.39,205.87);

\path[draw=drawColor,line width= 0.6pt,line join=round] ( 39.04,291.22) --
	(500.39,291.22);

\path[draw=drawColor,line width= 0.6pt,line join=round] ( 39.04,376.57) --
	(500.39,376.57);

\path[draw=drawColor,line width= 0.6pt,line join=round] ( 39.04,461.92) --
	(500.39,461.92);

\path[draw=drawColor,line width= 0.6pt,line join=round] (158.69, 29.52) --
	(158.69,500.39);

\path[draw=drawColor,line width= 0.6pt,line join=round] (208.03, 29.52) --
	(208.03,500.39);

\path[draw=drawColor,line width= 0.6pt,line join=round] (269.71, 29.52) --
	(269.71,500.39);
\definecolor{drawColor}{RGB}{0,0,0}
\definecolor{fillColor}{RGB}{0,0,0}

\path[draw=drawColor,line width= 0.4pt,line join=round,line cap=round,fill=fillColor] (380.73,427.78) circle (  1.96);

\path[draw=drawColor,line width= 0.4pt,line join=round,line cap=round,fill=fillColor] (282.05,111.99) circle (  1.96);

\path[draw=drawColor,line width= 0.4pt,line join=round,line cap=round,fill=fillColor] (306.72,240.01) circle (  1.96);

\path[draw=drawColor,line width= 0.4pt,line join=round,line cap=round,fill=fillColor] (306.72,368.03) circle (  1.96);

\path[draw=drawColor,line width= 0.4pt,line join=round,line cap=round,fill=fillColor] (158.69,137.59) circle (  1.96);

\path[draw=drawColor,line width= 0.4pt,line join=round,line cap=round,fill=fillColor] (282.05,308.29) circle (  1.96);

\path[draw=drawColor,line width= 0.4pt,line join=round,line cap=round,fill=fillColor] (380.73,316.82) circle (  1.96);

\path[draw=drawColor,line width= 0.4pt,line join=round,line cap=round,fill=fillColor] (479.42,478.99) circle (  1.96);

\path[draw=drawColor,line width= 0.4pt,line join=round,line cap=round,fill=fillColor] (183.36,171.73) circle (  1.96);

\path[draw=drawColor,line width= 0.4pt,line join=round,line cap=round,fill=fillColor] (306.72,376.57) circle (  1.96);

\path[draw=drawColor,line width= 0.4pt,line join=round,line cap=round,fill=fillColor] (282.05,308.29) circle (  1.96);

\path[draw=drawColor,line width= 0.4pt,line join=round,line cap=round,fill=fillColor] (232.71,265.61) circle (  1.96);

\path[draw=drawColor,line width= 0.4pt,line join=round,line cap=round,fill=fillColor] (257.38,368.03) circle (  1.96);

\path[draw=drawColor,line width= 0.4pt,line join=round,line cap=round,fill=fillColor] (134.02,111.99) circle (  1.96);

\path[draw=drawColor,line width= 0.4pt,line join=round,line cap=round,fill=fillColor] (356.06,368.03) circle (  1.96);

\path[draw=drawColor,line width= 0.4pt,line join=round,line cap=round,fill=fillColor] (158.69,111.99) circle (  1.96);

\path[draw=drawColor,line width= 0.4pt,line join=round,line cap=round,fill=fillColor] (380.73,402.17) circle (  1.96);

\path[draw=drawColor,line width= 0.4pt,line join=round,line cap=round,fill=fillColor] (257.38,163.20) circle (  1.96);

\path[draw=drawColor,line width= 0.4pt,line join=round,line cap=round,fill=fillColor] ( 60.01,129.06) circle (  1.96);

\path[draw=drawColor,line width= 0.4pt,line join=round,line cap=round,fill=fillColor] (208.03,120.52) circle (  1.96);
\definecolor{drawColor}{RGB}{190,190,190}

\path[draw=drawColor,line width= 1.1pt,line join=round] ( 60.01, 50.93) --
	( 65.32, 56.33) --
	( 70.62, 61.73) --
	( 75.93, 67.14) --
	( 81.24, 72.54) --
	( 86.55, 77.94) --
	( 91.86, 83.34) --
	( 97.17, 88.75) --
	(102.48, 94.15) --
	(107.79, 99.55) --
	(113.10,104.95) --
	(118.41,110.36) --
	(123.71,115.76) --
	(129.02,121.16) --
	(134.33,126.57) --
	(139.64,131.97) --
	(144.95,137.37) --
	(150.26,142.77) --
	(155.57,148.18) --
	(160.88,153.58) --
	(166.19,158.98) --
	(171.50,164.38) --
	(176.80,169.79) --
	(182.11,175.19) --
	(187.42,180.59) --
	(192.73,185.99) --
	(198.04,191.40) --
	(203.35,196.80) --
	(208.66,202.20) --
	(213.97,207.61) --
	(219.28,213.01) --
	(224.59,218.41) --
	(229.90,223.81) --
	(235.20,229.22) --
	(240.51,234.62) --
	(245.82,240.02) --
	(251.13,245.42) --
	(256.44,250.83) --
	(261.75,256.23) --
	(267.06,261.63) --
	(272.37,267.04) --
	(277.68,272.44) --
	(282.99,277.84) --
	(288.29,283.24) --
	(293.60,288.65) --
	(298.91,294.05) --
	(304.22,299.45) --
	(309.53,304.85) --
	(314.84,310.26) --
	(320.15,315.66) --
	(325.46,321.06) --
	(330.77,326.47) --
	(336.08,331.87) --
	(341.38,337.27) --
	(346.69,342.67) --
	(352.00,348.08) --
	(357.31,353.48) --
	(362.62,358.88) --
	(367.93,364.28) --
	(373.24,369.69) --
	(378.55,375.09) --
	(383.86,380.49) --
	(389.17,385.89) --
	(394.47,391.30) --
	(399.78,396.70) --
	(405.09,402.10) --
	(410.40,407.51) --
	(415.71,412.91) --
	(421.02,418.31) --
	(426.33,423.71) --
	(431.64,429.12) --
	(436.95,434.52) --
	(442.26,439.92) --
	(447.57,445.32) --
	(452.87,450.73) --
	(458.18,456.13) --
	(463.49,461.53) --
	(468.80,466.94) --
	(474.11,472.34) --
	(479.42,477.74);
\definecolor{drawColor}{gray}{0.20}

\path[draw=drawColor,line width= 0.6pt,line join=round,line cap=round] ( 39.04, 29.52) rectangle (500.39,500.39);
\end{scope}
\begin{scope}
\path[clip] (  0.00,  0.00) rectangle (505.89,505.89);
\definecolor{drawColor}{gray}{0.30}

\node[text=drawColor,anchor=base east,inner sep=0pt, outer sep=0pt, scale=  0.80] at ( 34.09, 32.16) {70};

\node[text=drawColor,anchor=base east,inner sep=0pt, outer sep=0pt, scale=  0.80] at ( 34.09,117.50) {80};

\node[text=drawColor,anchor=base east,inner sep=0pt, outer sep=0pt, scale=  0.80] at ( 34.09,202.85) {90};

\node[text=drawColor,anchor=base east,inner sep=0pt, outer sep=0pt, scale=  0.80] at ( 34.09,288.20) {100};

\node[text=drawColor,anchor=base east,inner sep=0pt, outer sep=0pt, scale=  0.80] at ( 34.09,373.55) {110};

\node[text=drawColor,anchor=base east,inner sep=0pt, outer sep=0pt, scale=  0.80] at ( 34.09,458.90) {120};
\end{scope}
\begin{scope}
\path[clip] (  0.00,  0.00) rectangle (505.89,505.89);
\definecolor{drawColor}{gray}{0.20}

\path[draw=drawColor,line width= 0.6pt,line join=round] ( 36.29, 35.17) --
	( 39.04, 35.17);

\path[draw=drawColor,line width= 0.6pt,line join=round] ( 36.29,120.52) --
	( 39.04,120.52);

\path[draw=drawColor,line width= 0.6pt,line join=round] ( 36.29,205.87) --
	( 39.04,205.87);

\path[draw=drawColor,line width= 0.6pt,line join=round] ( 36.29,291.22) --
	( 39.04,291.22);

\path[draw=drawColor,line width= 0.6pt,line join=round] ( 36.29,376.57) --
	( 39.04,376.57);

\path[draw=drawColor,line width= 0.6pt,line join=round] ( 36.29,461.92) --
	( 39.04,461.92);
\end{scope}
\begin{scope}
\path[clip] (  0.00,  0.00) rectangle (505.89,505.89);
\definecolor{drawColor}{gray}{0.20}

\path[draw=drawColor,line width= 0.6pt,line join=round] (158.69, 26.77) --
	(158.69, 29.52);

\path[draw=drawColor,line width= 0.6pt,line join=round] (208.03, 26.77) --
	(208.03, 29.52);

\path[draw=drawColor,line width= 0.6pt,line join=round] (269.71, 26.77) --
	(269.71, 29.52);
\end{scope}
\begin{scope}
\path[clip] (  0.00,  0.00) rectangle (505.89,505.89);
\definecolor{drawColor}{gray}{0.30}

\node[text=drawColor,anchor=base,inner sep=0pt, outer sep=0pt, scale=  0.80] at (158.69, 18.54) {$0$};

\node[text=drawColor,anchor=base,inner sep=0pt, outer sep=0pt, scale=  0.80] at (208.03, 18.54) {$x_{20}$};

\node[text=drawColor,anchor=base,inner sep=0pt, outer sep=0pt, scale=  0.80] at (269.71, 18.54) {$\bar x$};
\end{scope}
\begin{scope}
\path[clip] (  0.00,  0.00) rectangle (505.89,505.89);
\definecolor{drawColor}{RGB}{0,0,0}

\node[text=drawColor,rotate= 90.00,anchor=base,inner sep=0pt, outer sep=0pt, scale=  1.00] at ( 13.04,264.96) {$y_i$};
\end{scope}

\end{tikzpicture}
\end{figure}
\end{center}

\begin{enumerate}
\item Изобразите на плоскости $\hat y_{20}$, $y_{20}$, $\bar y$, $\hat u_{20}$, $\hb_1$.
\item Что произойдёт с каждой из указанных величин при добавлении нового наблюдения равного $(\bar x + 1, \bar y + \hb_2)$?
\end{enumerate}




\newpage
\item По квартальным данным 1958-1976 годов была оценена модель с тремя объясняющими факторами:
\[
\hat Y_i = 2.2 + 0.104 X_i - 3.48 Z_i + 0.34 W_i, \; ESS = 100, \; RSS = 20
\]

\begin{enumerate}
\item Какую модель необходимо оценить исследователю, если он считает, что в различные сезоны среднее значение зависимой переменной помимо зависимости от трёх регрессоров может отличаться на константу?
\item При оценивании модели, допускающей сезонные эффекты, оказалось, что значение $ESS$ увеличилось до $170$.
На уровне значимости 5\% проверьте гипотезу о наличии сезонности.
\end{enumerate}



\item По 24 наблюдениям была оценена модель:

\[
\widehat{Y}_i=15-4Z_i+3W_i
\]

Известно, что случайные ошибки нормально распределены, $RSS=180$, и

\[
(X'X)^{-1} =
\ensuremath{% latex table generated in R 3.3.2 by xtable 1.8-2 package
% Mon Jan 16 07:59:08 2017
\begin{pmatrix}{}
  0.171 & -0.073 & -0.066 \\ 
  -0.073 & 0.102 & -0.016 \\ 
  -0.066 & -0.016 & 0.081 \\ 
  \end{pmatrix}
}
\]


\begin{enumerate}
\item Проверьте гипотезу $H_0: \beta_Z = 0$ против $H_a: \beta_Z \neq 0$ на уровне значимости~5\%.
\item Проверьте гипотезу $H_0: \beta_Z - \beta_W = 0$  против $H_a: \beta_Z - \beta_W \neq 0$ на уровне значимости~5\%.
\item Выпишите использованные при проверке гипотез предпосылки о случайных ошибках модели.
\end{enumerate}


\end{enumerate}

\end{document}
