\documentclass[12pt]{article}\usepackage[]{graphicx}\usepackage[svgnames]{xcolor}
%% maxwidth is the original width if it is less than linewidth
%% otherwise use linewidth (to make sure the graphics do not exceed the margin)
\makeatletter
\def\maxwidth{ %
  \ifdim\Gin@nat@width>\linewidth
    \linewidth
  \else
    \Gin@nat@width
  \fi
}
\makeatother

\definecolor{fgcolor}{rgb}{0.345, 0.345, 0.345}
\newcommand{\hlnum}[1]{\textcolor[rgb]{0.686,0.059,0.569}{#1}}%
\newcommand{\hlstr}[1]{\textcolor[rgb]{0.192,0.494,0.8}{#1}}%
\newcommand{\hlcom}[1]{\textcolor[rgb]{0.678,0.584,0.686}{\textit{#1}}}%
\newcommand{\hlopt}[1]{\textcolor[rgb]{0,0,0}{#1}}%
\newcommand{\hlstd}[1]{\textcolor[rgb]{0.345,0.345,0.345}{#1}}%
\newcommand{\hlkwa}[1]{\textcolor[rgb]{0.161,0.373,0.58}{\textbf{#1}}}%
\newcommand{\hlkwb}[1]{\textcolor[rgb]{0.69,0.353,0.396}{#1}}%
\newcommand{\hlkwc}[1]{\textcolor[rgb]{0.333,0.667,0.333}{#1}}%
\newcommand{\hlkwd}[1]{\textcolor[rgb]{0.737,0.353,0.396}{\textbf{#1}}}%
\let\hlipl\hlkwb

\usepackage{framed}
\makeatletter
\newenvironment{kframe}{%
 \def\at@end@of@kframe{}%
 \ifinner\ifhmode%
  \def\at@end@of@kframe{\end{minipage}}%
  \begin{minipage}{\columnwidth}%
 \fi\fi%
 \def\FrameCommand##1{\hskip\@totalleftmargin \hskip-\fboxsep
 \colorbox{shadecolor}{##1}\hskip-\fboxsep
     % There is no \\@totalrightmargin, so:
     \hskip-\linewidth \hskip-\@totalleftmargin \hskip\columnwidth}%
 \MakeFramed {\advance\hsize-\width
   \@totalleftmargin\z@ \linewidth\hsize
   \@setminipage}}%
 {\par\unskip\endMakeFramed%
 \at@end@of@kframe}
\makeatother

\definecolor{shadecolor}{rgb}{.97, .97, .97}
\definecolor{messagecolor}{rgb}{0, 0, 0}
\definecolor{warningcolor}{rgb}{1, 0, 1}
\definecolor{errorcolor}{rgb}{1, 0, 0}
\newenvironment{knitrout}{}{} % an empty environment to be redefined in TeX

\usepackage{alltt}



\usepackage[top=2cm, left=1cm, right=1cm, bottom=2cm]{geometry} % размер текста на странице

\usepackage[box, % запрет на перенос вопросов
%nopage,
insidebox, % ставим буквы в квадратики
separateanswersheet, % добавляем бланк ответов
nowatermark, % отсутствие надписи "Черновик"
% indivanswers,  % показываем верные ответы
% answers, % листок с ответами
lang=RU,
nopage, % убираем оформление страницы (идентификаторы для распознавания)
completemulti]{automultiplechoice}

\usepackage{floatrow}

\usepackage{tikz} % картинки в tikz
\usepackage{microtype} % свешивание пунктуации

\usepackage{array} % для столбцов фиксированной ширины
\usepackage{url}

\usepackage{indentfirst} % отступ в первом параграфе

\usepackage{sectsty} % для центрирования названий частей
\allsectionsfont{\centering}

\usepackage{amsmath, amsfonts} % куча стандартных математических плюшек

\usepackage{multicol} % текст в несколько колонок

\usepackage{lastpage} % чтобы узнать номер последней страницы

\usepackage{caption} % для пустых подписей и прочих плюшек

\usepackage{enumitem} % дополнительные плюшки для списков
%  например \begin{enumerate}[resume] позволяет продолжить нумерацию в новом списке










\usepackage{fancyhdr} % весёлые колонтитулы
\pagestyle{fancy}
\lhead{Эконометрика, контрольная 1}
\chead{}
\rhead{24.12.2016, переписывание}
\lfoot{}
\cfoot{}
\rfoot{\thepage/\pageref{LastPage}}
\renewcommand{\headrulewidth}{0.4pt}
\renewcommand{\footrulewidth}{0.4pt}



\usepackage{todonotes} % для вставки в документ заметок о том, что осталось сделать
% \todo{Здесь надо коэффициенты исправить}
% \missingfigure{Здесь будет Последний день Помпеи}
% \listoftodos --- печатает все поставленные \todo'шки


% более красивые таблицы
\usepackage{booktabs}
% заповеди из докупентации:
% 1. Не используйте вертикальные линни
% 2. Не используйте двойные линии
% 3. Единицы измерения - в шапку таблицы
% 4. Не сокращайте .1 вместо 0.1
% 5. Повторяющееся значение повторяйте, а не говорите "то же"



\usepackage{fontspec}
\usepackage{polyglossia}

\setmainlanguage{russian}
\setotherlanguages{english}

% download "Linux Libertine" fonts:
% http://www.linuxlibertine.org/index.php?id=91&L=1
\setmainfont{Linux Libertine O} % or Helvetica, Arial, Cambria
% why do we need \newfontfamily:
% http://tex.stackexchange.com/questions/91507/
\newfontfamily{\cyrillicfonttt}{Linux Libertine O}

\AddEnumerateCounter{\asbuk}{\russian@alph}{щ} % для списков с русскими буквами


%% эконометрические сокращения
\DeclareMathOperator{\plim}{plim}
\DeclareMathOperator{\Cov}{Cov}
\DeclareMathOperator{\Corr}{Corr}
\DeclareMathOperator{\Var}{Var}
\DeclareMathOperator{\E}{E}
\def \hb{\hat{\beta}}
\def \hs{\hat{\sigma}}
\def \htheta{\hat{\theta}}
\def \s{\sigma}
\def \hy{\hat{y}}
\def \hY{\hat{Y}}
\def \v1{\vec{1}}
\def \e{\varepsilon}
\def \he{\hat{\e}}
\def \z{z}
\def \hVar{\widehat{\Var}}
\def \hCorr{\widehat{\Corr}}
\def \hCov{\widehat{\Cov}}
\def \cN{\mathcal{N}}


\AddEnumerateCounter{\asbuk}{\russian@alph}{щ} % для списков с русскими буквами
\setlist[enumerate, 2]{label=\asbuk*),ref=\asbuk*}
\IfFileExists{upquote.sty}{\usepackage{upquote}}{}
\begin{document}


\element{combat_kr_01_16_retake}{ % в фигурных скобках название группы вопросов
 %\AMCnoCompleteMulti
  \begin{questionmult}{pr_01} % тип вопроса (questionmult --- множественный выбор) и в фигурных --- номер вопроса
    Если $\E(X) = 5$, $\E(Y) = 4$, $\Var(X) = 6$, $\Var(Y) = 7$, $\Cov(X,Y) = -1$, то $\Cov(2 - X + 2Y, 2X)$  равна
 \begin{multicols}{3} % располагаем ответы в 3 колонки
   \begin{choices} % опция [o] не рандомизирует порядок ответов
      \correctchoice{-16}
      \wrongchoice{16}
      \wrongchoice{8}
      \wrongchoice{-8}
      \wrongchoice{-12}
      \wrongchoice{-20}
   \end{choices}
  \end{multicols}
  \end{questionmult}
}

\element{combat_kr_01_16_retake}{ % в фигурных скобках название группы вопросов
 %\AMCnoCompleteMulti
  \begin{questionmult}{pr_02} % тип вопроса (questionmult --- множественный выбор) и в фигурных --- номер вопроса
    Джеймс Бонд оценил парную регрессию и оказалось, что $\hat Y_i = 5 + 6 X_i$. Если Джеймс Бонд оценит регрессию без константы, то окажется, что
 \begin{multicols}{3} % располагаем ответы в 3 колонки
   \begin{choices} % опция [o] не рандомизирует порядок ответов
      %\correctchoice{}
      \wrongchoice{$\hat Y_i = 6 X_i$}
      \wrongchoice{$\hat Y_i = 11 X_i$}
      \wrongchoice{$\hat Y_i = 5$}
      \wrongchoice{$\hat Y_i = 11$}
      \wrongchoice{$\hat Y_i = 5.5 X_i$}
      \wrongchoice{$\hat Y_i = 5.5$}
      \wrongchoice{недостаточно информации для оценивания}
   \end{choices}
  \end{multicols}
  \end{questionmult}
}



\element{combat_kr_01_16_retake}{ % в фигурных скобках название группы вопросов
 %\AMCnoCompleteMulti
  \begin{questionmult}{pr_03} % тип вопроса (questionmult --- множественный выбор) и в фигурных --- номер вопроса
    Предпосылки теоремы Гаусса-Маркова выполнены, случайные ошибки нормально распределены, уровень доверия равен $90$\%, критическое значение $t$-статистики равно $2.35$, всего $n$ наблюдений. Регрессия имеет вид $\hat Y_i = \underset{(3)}{-4} + \underset{(0.2)}{5}X_i$, в скобках указаны стандартные ошибки. Доверительный интервал для $\beta_2$ равен
 \begin{multicols}{3} % располагаем ответы в 3 колонки
   \begin{choices} % опция [o] не рандомизирует порядок ответов
      \correctchoice{$[4.53; 5.47]$}
      \wrongchoice{$[2.65; 7.35]$}
      \wrongchoice{$[0.3; 9.7]$}
      \wrongchoice{$[4.79; 5.21]$}
      \wrongchoice{$[3.95; 6.05]$}
%      \wrongchoice{не существует}
   \end{choices}
  \end{multicols}
  \end{questionmult}
}

\element{combat_kr_01_16}{ % в фигурных скобках название группы вопросов
 %\AMCnoCompleteMulti
  \begin{questionmult}{pr_04} % тип вопроса (questionmult --- множественный выбор) и в фигурных --- номер вопроса
    Имеются данные по доходу жены, мужа и продолжительности брака. Доход семьи складывается из дохода жены и мужа. Вася оценил зависимость дохода семьи от продолжительности брака и получил регрессию $\hat Y_i  = 20 + 3X_i$, Петя оценил зависимость дохода мужа от продолжительности брака и получил регрессию $\hat Y_i  = 10 + 2X_i$. Маша оценивает зависимость дохода жены от продолжительности брака. Она получит регрессию:
 \begin{multicols}{2} % располагаем ответы в 3 колонки
   \begin{choices} % опция [o] не рандомизирует порядок ответов
      \correctchoice{$\hat Y_i  = 10+X_i$}
      \wrongchoice{$\hat Y_i  = 10-X_i$}
      \wrongchoice{$\hat Y_i  = 30+5X_i$}
      \wrongchoice{$\hat Y_i  = 20+3X_i$}
      \wrongchoice{$\hat Y_i  = 15+2.5X_i$}
      \wrongchoice{недостаточно данных для ответа}
   \end{choices}
  \end{multicols}
  \end{questionmult}
}

\element{combat_kr_01_16}{ % в фигурных скобках название группы вопросов
 %\AMCnoCompleteMulti
  \begin{questionmult}{pr_05} % тип вопроса (questionmult --- множественный выбор) и в фигурных --- номер вопроса
    В парной регрессии на уровне значимости 5\%-ов гипотеза $H_0$: $\beta_2 = 2016$ не отвергается. Из этого можно сделать вывод, что на соответствующем уровне значимости
 \begin{multicols}{2} % располагаем ответы в 3 колонки
   \begin{choices} % опция [o] не рандомизирует порядок ответов
      \wrongchoice{$H_0$: $\beta_2 = 0$ отвергается}
      \wrongchoice{$H_0$: $\beta_2 = 0$ не отвергается}
      \wrongchoice{$H_a$: $\beta_2 \neq 0$ не отвергается}
      \wrongchoice{$H_a$: $\beta_2 \neq 0$ отвергается}
      \wrongchoice{доверительный интервал для $\beta_2$ не содержит ноль}
   \end{choices}
  \end{multicols}
  \end{questionmult}
}


\element{combat_kr_01_16_retake}{ % в фигурных скобках название группы вопросов
 %\AMCnoCompleteMulti
  \begin{questionmult}{pr_06} % тип вопроса (questionmult --- множественный выбор) и в фигурных --- номер вопроса
    В парной регрессии величина $\bar Y - \hat\beta_1 - \hat\beta_2 \bar X$
 \begin{multicols}{2} % располагаем ответы в 3 колонки
   \begin{choices} % опция [o] не рандомизирует порядок ответов
      \correctchoice{равна 0}
      \wrongchoice{равна 1}
      \wrongchoice{равна (-1)}
      \wrongchoice{не существует}
      \wrongchoice{может принимать любое неотрицательное значение}
      \wrongchoice{может принимать любое положительное значение}
   \end{choices}
  \end{multicols}
  \end{questionmult}
}



\element{combat_kr_01_16}{ % в фигурных скобках название группы вопросов
 %\AMCnoCompleteMulti
  \begin{questionmult}{pr_07} % тип вопроса (questionmult --- множественный выбор) и в фигурных --- номер вопроса
    Условием теоремы Гаусса-Маркова, необходимым для несмещённости оценок коэффициентов регрессии в модели $Y_i = \beta_1 + \beta_2 X_i + u_i$ является
 \begin{multicols}{2} % располагаем ответы в 3 колонки
   \begin{choices} % опция [o] не рандомизирует порядок ответов
      \correctchoice{$\E(u_i)=0$}
      \wrongchoice{гомоскедастичность случайных ошибок}
      \wrongchoice{гетероскедастичность случайных ошибок}
      \wrongchoice{нормальность случайных ошибок}
      \wrongchoice{некоррелированность случайных ошибок}
      \wrongchoice{$\E(u_i)\neq 0$}
   \end{choices}
  \end{multicols}
  \end{questionmult}
}




\element{combat_kr_01_16_retake}{ % в фигурных скобках название группы вопросов
 %\AMCnoCompleteMulti
  \begin{questionmult}{pr_08} % тип вопроса (questionmult --- множественный выбор) и в фигурных --- номер вопроса
    В модели парной регрессии $R^2 = 0.9$, $TSS = 300$ и $12$ наблюдений. Несмещённая оценка дисперсии случайной ошибки равна
 \begin{multicols}{3} % располагаем ответы в 3 колонки
   \begin{choices} % опция [o] не рандомизирует порядок ответов
      \correctchoice{$3$}
      \wrongchoice{$3.1$}
      \wrongchoice{$3.2$}
      \wrongchoice{$2.8$}
      \wrongchoice{$2.9$}
      \wrongchoice{$3.3$}
   \end{choices}
  \end{multicols}
  \end{questionmult}
}





\element{combat_kr_01_16_retake}{ % в фигурных скобках название группы вопросов
 %\AMCnoCompleteMulti
  \begin{questionmult}{pr_09} % тип вопроса (questionmult --- множественный выбор) и в фигурных --- номер вопроса
    Если все $Y_i$ в линейной регрессии увеличить в два раза, то оценка $\hat\beta_2$
 \begin{multicols}{2} % располагаем ответы в 3 колонки
   \begin{choices} % опция [o] не рандомизирует порядок ответов
      \correctchoice{помножится на $2$}
      \wrongchoice{помножится на $4$}
      \wrongchoice{поделится на $2$}
      \wrongchoice{поделится на $4$}
      \wrongchoice{не изменится}
      \wrongchoice{изменится в произвольную сторону, в зависимости от $X_i$}
   \end{choices}
  \end{multicols}
  \end{questionmult}
}






\element{combat_kr_01_16_retake}{ % в фигурных скобках название группы вопросов
 %\AMCnoCompleteMulti
  \begin{questionmult}{pr_10} % тип вопроса (questionmult --- множественный выбор) и в фигурных --- номер вопроса
   Если $\alpha = 0.05$ и $P$-значение равно $0.04$, то
 \begin{multicols}{2} % располагаем ответы в 3 колонки
   \begin{choices} % опция [o] не рандомизирует порядок ответов
      \correctchoice{$H_0$ отвергается}
      \wrongchoice{$H_0$ принимается}
      \wrongchoice{$H_a$ отвергается}
      \wrongchoice{$H_a$ не отвергается}
      \wrongchoice{$H_a$ принимается}
      \wrongchoice{недостаточно информации для ответа}
   \end{choices}
  \end{multicols}
  \end{questionmult}
}



\element{combat_kr_01_16}{ % в фигурных скобках название группы вопросов
 %\AMCnoCompleteMulti
  \begin{questionmult}{pr_11} % тип вопроса (questionmult --- множественный выбор) и в фигурных --- номер вопроса
    Свободно распространяемым программным обеспечением является
 \begin{multicols}{3} % располагаем ответы в 3 колонки
   \begin{choices} % опция [o] не рандомизирует порядок ответов
      \correctchoice{R}
      \wrongchoice{Excel}
      \wrongchoice{Stata}
      \wrongchoice{Eviews}
      \wrongchoice{SPSS}
      \wrongchoice{Matlab}
   \end{choices}
  \end{multicols}
  \end{questionmult}
}


%%% демо и всякая фигня с прошлого







\element{demo_kr_01_16}{ % в фигурных скобках название группы вопросов
 %\AMCnoCompleteMulti
  \begin{questionmult}{pr1} % тип вопроса (questionmult --- множественный выбор) и в фигурных --- номер вопроса
    Если $\E(X) = -3$, $\E(Y) = 2$, $\Var(X) = 6$, $\Var(Y) = 7$, $\Cov(X,Y) = -1$, то $\Var(5X + 2Y-1)$  равна
 \begin{multicols}{3} % располагаем ответы в 3 колонки
   \begin{choices} % опция [o] не рандомизирует порядок ответов
      \correctchoice{158}
      \wrongchoice{178}
      \wrongchoice{198}
      \wrongchoice{148}
      \wrongchoice{168}
      \wrongchoice{169}
   \end{choices}
  \end{multicols}
  \end{questionmult}
}

\element{combat_kr_01_16_retake}{ % в фигурных скобках название группы вопросов
 % \AMCnoCompleteMulti
  \begin{questionmult}{1} % тип вопроса (questionmult --- множественный выбор) и в фигурных --- номер вопроса
  При добавлении нового наблюдения
% \begin{multicols}{3} % располагаем ответы в 3 колонки
   \begin{choices} % опция [o] не рандомизирует порядок ответов
      \correctchoice{$RSS$ не уменьшится; $TSS$ не уменьшится}
      \wrongchoice{$RSS$ может измениться произвольно; $TSS$ не уменьшится}
      \wrongchoice{$RSS$ не увеличится; $TSS$ не уменьшится}
      \wrongchoice{$RSS$ может измениться произвольно; $TSS$ не увеличиться}
      \wrongchoice{$RSS$ может измениться произвольно; $TSS$ может измениться произвольно}
   \end{choices}
%  \end{multicols}
  \end{questionmult}
}


\element{demo_kr_01_16}{ % в фигурных скобках название группы вопросов
 \AMCnoCompleteMulti
  \begin{questionmult}{2} % тип вопроса (questionmult --- множественный выбор) и в фигурных --- номер вопроса
  Если в модели парной регрессии $Y_i=\beta_1 + \beta_2 X_i + u_i$ все $X_i$ равны константе $2016$, то оценка $\hat \beta_2$ равна
 \begin{multicols}{3} % располагаем ответы в 3 колонки
   \begin{choices} % опция [o] не рандомизирует порядок ответов
      \wrongchoice{$2016$}
      \wrongchoice{$1/2016$}
      \wrongchoice{$-2016$}
      \wrongchoice{$-1/2016$}
      \wrongchoice{$0$}
      \correctchoice{не существует}
      \end{choices}
  \end{multicols}
  \end{questionmult}
}

\element{combat_kr_01_16_retake}{ % в фигурных скобках название группы вопросов
 % \AMCnoCompleteMulti
  \begin{questionmult}{3} % тип вопроса (questionmult --- множественный выбор) и в фигурных --- номер вопроса
  Если при оценке парной регрессии оказалось, что для любого наблюдения $\hat Y_i < Y_i$, то  
 \begin{multicols}{3} % располагаем ответы в 3 колонки
   \begin{choices} % опция [o] не рандомизирует порядок ответов
      \wrongchoice{$\hb_1 < 0$}
      \wrongchoice{$\hb_2 < 0$}
      \wrongchoice{$\hb_1 > 0$}
      \wrongchoice{$\hb_2 > 0$}
      \correctchoice{такое невозможно}
      \end{choices}
  \end{multicols}
  \end{questionmult}
}

\element{demo_kr_01_16}{ % в фигурных скобках название группы вопросов
 %\AMCnoCompleteMulti
  \begin{questionmult}{4} % тип вопроса (questionmult --- множественный выбор) и в фигурных --- номер вопроса
  Квартальные данные о ВВП России за 10 лет являются
 \begin{multicols}{2} % располагаем ответы в 3 колонки
   \begin{choices} % опция [o] не рандомизирует порядок ответов
      \wrongchoice{панельными данными}
      \wrongchoice{перекрестной выборкой}
      \wrongchoice{случайной выборкой}
      \wrongchoice{сходящимся рядом}
      \correctchoice{временным рядом}
      \end{choices}
  \end{multicols}
  \end{questionmult}
}

\element{demo_kr_01_16}{ % в фигурных скобках название группы вопросов
 \AMCnoCompleteMulti
  \begin{questionmult}{5} % тип вопроса (questionmult --- множественный выбор) и в фигурных --- номер вопроса
Предпосылки теоремы Гаусса-Маркова выполнены, случайные ошибки нормально распределены. Регрессия по 25 наблюдениям имеет вид $\hat Y_i = \underset{(2)}{-1}  + \underset{(0.1)}{4}X_i$. В скобках указаны стандартные ошибки. На уровне значимости $0.05$
 \begin{multicols}{1} % располагаем ответы в 3 колонки
   \begin{choices} % опция [o] не рандомизирует порядок ответов
      \wrongchoice{оба коэффициента значимы}
      \correctchoice{значим только коэффициент наклона}
      \wrongchoice{оба коэффициента незначимы}
      \wrongchoice{недостаточно информации для определения значимости}
      \wrongchoice{значим только свободный член}
      \end{choices}
  \end{multicols}
  \end{questionmult}
}

\element{demo_kr_01_16}{ % в фигурных скобках название группы вопросов
 %\AMCnoCompleteMulti
  \begin{questionmult}{6} % тип вопроса (questionmult --- множественный выбор) и в фигурных --- номер вопроса
  Если $P$-значение $t$-статистики при проверке значимости коэффициента регрессии равно $0.04$, то этот коэффициент не значим при уровне значимости
 \begin{multicols}{3} % располагаем ответы в 3 колонки
   \begin{choices} % опция [o] не рандомизирует порядок ответов
      \wrongchoice{$0.05$}
      \wrongchoice{$0.95$}
      \wrongchoice{$0.9$}
      \wrongchoice{$0.1$}
      \correctchoice{$0.01$}
      \end{choices}
  \end{multicols}
  \end{questionmult}
}


\element{demo_kr_01_16}{ % в фигурных скобках название группы вопросов
 %\AMCnoCompleteMulti
  \begin{questionmult}{7} % тип вопроса (questionmult --- множественный выбор) и в фигурных --- номер вопроса
  Регрессия по 25 наблюдениям имеет вид $\hat Y_i = \underset{(2)}{-1}  - \underset{(0.5)}{1.5}X_i$. В скобках указаны стандартные ошибки. При проверке гипотезы о равенстве коэффициента наклона $(-1)$ расчётное значение $t$-статистики равно
 \begin{multicols}{3} % располагаем ответы в 3 колонки
   \begin{choices} % опция [o] не рандомизирует порядок ответов
      \wrongchoice{$-2$}
      \wrongchoice{$0.5$}
      \wrongchoice{$2$}
      \wrongchoice{$-0.5$}
      \correctchoice{$-1$}
      \end{choices}
  \end{multicols}
  \end{questionmult}
}


\element{demo_kr_01_16}{ % в фигурных скобках название группы вопросов
 \AMCnoCompleteMulti
  \begin{questionmult}{8} % тип вопроса (questionmult --- множественный выбор) и в фигурных --- номер вопроса
  В регрессии с константой, оценённой с помощью МНК, сумма остатков
 \begin{multicols}{2} % располагаем ответы в 3 колонки
   \begin{choices} % опция [o] не рандомизирует порядок ответов
      \wrongchoice{может принимать любое неположительное значение}
      \wrongchoice{может принимать любое положительное значение}
      \wrongchoice{может принимать любое значение из $\mathbb{R}$}
      \wrongchoice{равна $1$}
      \correctchoice{равна $0$}
      \wrongchoice{не существует}
      \end{choices}
  \end{multicols}
  \end{questionmult}
}


\element{combat_kr_01_16_retake}{ % в фигурных скобках название группы вопросов
 % \AMCnoCompleteMulti
  \begin{questionmult}{9} % тип вопроса (questionmult --- множественный выбор) и в фигурных --- номер вопроса
  Если в регрессии с константой, оценённой с помощью МНК, сумма квадратов остатков равна нулю, то показатель $R^2$
 \begin{multicols}{2} % располагаем ответы в 3 колонки
   \begin{choices} % опция [o] не рандомизирует порядок ответов
      \wrongchoice{равен $0$}
      \wrongchoice{равен $-1$}
      \wrongchoice{не существует}
      \wrongchoice{может принимать любое значение на $[0;1]$}
      \correctchoice{равен $1$}
      \wrongchoice{не существует}
      \end{choices}
  \end{multicols}
  \end{questionmult}
}

\element{demo_kr_01_16}{ % в фигурных скобках название группы вопросов
 %\AMCnoCompleteMulti
  \begin{questionmult}{10} % тип вопроса (questionmult --- множественный выбор) и в фигурных --- номер вопроса
  Необходимым условием теоремы Гаусса-Маркова является
 \begin{multicols}{2} % располагаем ответы в 3 колонки
   \begin{choices} % опция [o] не рандомизирует порядок ответов
      \wrongchoice{наличие в матрице $X$ единичного столбца}
      \wrongchoice{нормальность остатков}
      \wrongchoice{нормальность $Y_i$}
      \wrongchoice{постоянство дисперсии остатков}
      \correctchoice{постоянство дисперсии случайной ошибки}
      \end{choices}
  \end{multicols}
  \end{questionmult}
}




\element{demo_15}{ % в фигурных скобках название группы вопросов
 \AMCnoCompleteMulti
  \begin{questionmult}{1} % тип вопроса (questionmult --- множественный выбор) и в фигурных --- номер вопроса
  При добавлении новой переменной скорректированный $R^2$
% \begin{multicols}{3} % располагаем ответы в 3 колонки
   \begin{choices} % опция [o] не рандомизирует порядок ответов
      \wrongchoice{обязательно вырастет}
      \wrongchoice{обязательно упадёт}
       \correctchoice{может как вырасти, так и упасть}
      \end{choices}
%  \end{multicols}
  \end{questionmult}
}


\element{demo_15}{ % в фигурных скобках название группы вопросов
 \AMCnoCompleteMulti
  \begin{questionmult}{2} % тип вопроса (questionmult --- множественный выбор) и в фигурных --- номер вопроса
  При добавлении новой переменной коэффициент детерминации $R^2$:
% \begin{multicols}{3} % располагаем ответы в 3 колонки
   \begin{choices} % опция [o] не рандомизирует порядок ответов
      \correctchoice{обязательно вырастет}
      \wrongchoice{обязательно упадёт}
      \wrongchoice{может как вырасти, так и упасть}
      \end{choices}
%  \end{multicols}
  \end{questionmult}
}


\element{demo_15}{ % в фигурных скобках название группы вопросов
 \AMCnoCompleteMulti
  \begin{questionmult}{3} % тип вопроса (questionmult --- множественный выбор) и в фигурных --- номер вопроса
  Для проверки гипотезы о значимости коэффициентов при мультиколлинеарности стандартные $t$-статистики
% \begin{multicols}{3} % располагаем ответы в 3 колонки
   \begin{choices} % опция [o] не рандомизирует порядок ответов
      \correctchoice{можно использовать, т.к. они по прежнему имеют $t$-распределение}
      \wrongchoice{нельзя использовать т.к. они не имеют $t$-распределения}
      \end{choices}
%  \end{multicols}
  \end{questionmult}
}

\element{demo_15}{ % в фигурных скобках название группы вопросов
 \AMCnoCompleteMulti
  \begin{questionmult}{4} % тип вопроса (questionmult --- множественный выбор) и в фигурных --- номер вопроса
  При условной гетероскедастичности и наблюдениях, представляющих случайную выборку, оценки МНК
 \begin{multicols}{2} % располагаем ответы в 3 колонки
   \begin{choices} % опция [o] не рандомизирует порядок ответов
      \correctchoice{остаются состоятельными}
      \wrongchoice{перестают быть состоятельными}
      \end{choices}
  \end{multicols}
  \end{questionmult}
}


\element{demo_15}{ % в фигурных скобках название группы вопросов
 \AMCnoCompleteMulti
  \begin{questionmult}{5} % тип вопроса (questionmult --- множественный выбор) и в фигурных --- номер вопроса
  При условной гетероскедастичности и наблюдениях, представляющих случайную выборку, оценки МНК
 \begin{multicols}{2} % располагаем ответы в 3 колонки
   \begin{choices} % опция [o] не рандомизирует порядок ответов
      \correctchoice{остаются несмещёнными}
      \wrongchoice{перестают быть несмещёнными}
      \end{choices}
  \end{multicols}
  \end{questionmult}
}


\element{demo_15}{ % в фигурных скобках название группы вопросов
 \AMCnoCompleteMulti
  \begin{questionmult}{6} % тип вопроса (questionmult --- множественный выбор) и в фигурных --- номер вопроса
  При предпосылке о нормально распределенных ошибках в классической линейной регрессионной модели оценки коэффициентов уравнения с помощью МНК и оценки с помощью максимального правдоподобия
% \begin{multicols}{3} % располагаем ответы в 3 колонки
   \begin{choices} % опция [o] не рандомизирует порядок ответов
      \correctchoice{совпадают}
      \wrongchoice{отличаются}
      \end{choices}
%  \end{multicols}
  \end{questionmult}
}

\element{demo_15}{ % в фигурных скобках название группы вопросов
 \AMCcompleteMulti
  \begin{questionmult}{7} % тип вопроса (questionmult --- множественный выбор) и в фигурных --- номер вопроса
  При условной гетероскедастичности использование робастных стандартных ошибок позволяет
% \begin{multicols}{3} % располагаем ответы в 3 колонки
   \begin{choices} % опция [o] не рандомизирует порядок ответов
      \wrongchoice{устранить смещённость оценок коэффициентов}
      \wrongchoice{устранить несостоятельность оценок коэффициентов}
      \end{choices}
%  \end{multicols}
  \end{questionmult}
}

\element{demo_15}{ % в фигурных скобках название группы вопросов
 \AMCcompleteMulti
  \begin{questionmult}{8} % тип вопроса (questionmult --- множественный выбор) и в фигурных --- номер вопроса
  При автокорреляции первого порядка в ошибках использование робастных стандартных ошибок Нью-Веста позволяет
% \begin{multicols}{3} % располагаем ответы в 3 колонки
   \begin{choices} % опция [o] не рандомизирует порядок ответов
     \wrongchoice{устранить смещённость оценок коэффициентов}
     \wrongchoice{устранить несостоятельность оценок коэффициентов}
      \end{choices}
%  \end{multicols}
  \end{questionmult}
}

\element{demo_15}{ % в фигурных скобках название группы вопросов
 \AMCnoCompleteMulti
  \begin{questionmult}{9} % тип вопроса (questionmult --- множественный выбор) и в фигурных --- номер вопроса
  Если нарушена только предпосылка $\E(u_i) = 0$, то при оценке модели $y_i = \beta_1 + \beta_2 x_i + u_i$ оценка $\hat \beta_2$ окажется
 \begin{multicols}{2} % располагаем ответы в 3 колонки
   \begin{choices} % опция [o] не рандомизирует порядок ответов
      \correctchoice{несмещённой}
      \wrongchoice{смещённой}
      \end{choices}
 \end{multicols}
  \end{questionmult}
}


\element{demo_15}{ % в фигурных скобках название группы вопросов
 \AMCnoCompleteMulti
  \begin{questionmult}{10} % тип вопроса (questionmult --- множественный выбор) и в фигурных --- номер вопроса
  Если все выборочные корреляции между регрессорами по модулю меньше 0.1 то строгая мультиколлинеарность
 \begin{multicols}{2} % располагаем ответы в 3 колонки
   \begin{choices} % опция [o] не рандомизирует порядок ответов
      \correctchoice{возможна}
      \wrongchoice{невозможна}
      \end{choices}
 \end{multicols}
  \end{questionmult}
}




















\element{combat_15}{ % в фигурных скобках название группы вопросов
 \AMCcompleteMulti
  \begin{questionmult}{1} % тип вопроса (questionmult --- множественный выбор) и в фигурных --- номер вопроса
  Предпосылка об отсутствии систематической ошибки в модели означает, что для всех наблюдений
 \begin{multicols}{2} % располагаем ответы в 3 колонки
   \begin{choices} % опция [o] не рандомизирует порядок ответов
      \wrongchoice{$\Var(\varepsilon_i)=0$}
      \wrongchoice{$\Var(\varepsilon_i) \neq 0$}
      \end{choices}
  \end{multicols}
  \end{questionmult}
}

\element{combat_15}{ % в фигурных скобках название группы вопросов
 \AMCnoCompleteMulti
  \begin{questionmult}{2} % тип вопроса (questionmult --- множественный выбор) и в фигурных --- номер вопроса
  Стандартные ошибки в форме Уайта в случае гетероскедастичности помогают устранить несостоятельность оценок коэффициентов
 \begin{multicols}{2} % располагаем ответы в 3 колонки
   \begin{choices} % опция [o] не рандомизирует порядок ответов
      \correctchoice{неверно}
      \wrongchoice{верно}
      \end{choices}
  \end{multicols}
  \end{questionmult}
}

\element{combat_15}{ % в фигурных скобках название группы вопросов
 \AMCnoCompleteMulti
  \begin{questionmult}{3} % тип вопроса (questionmult --- множественный выбор) и в фигурных --- номер вопроса
  Незначимость всех коэффициентов регрессии
 \begin{multicols}{2} % располагаем ответы в 3 колонки
   \begin{choices} % опция [o] не рандомизирует порядок ответов
      \correctchoice{может быть не связана с мультиколлинеарностью}
      \wrongchoice{обязательно свидетельствует о наличии мультиколлинеарности}
      \end{choices}
  \end{multicols}
  \end{questionmult}
}


\element{combat_15}{ % в фигурных скобках название группы вопросов
 \AMCnoCompleteMulti
  \begin{questionmult}{4} % тип вопроса (questionmult --- множественный выбор) и в фигурных --- номер вопроса
  После применения МНК к модели $y_i=\beta x_i + \varepsilon_i$   сумма $ESS+RSS$
 \begin{multicols}{2} % располагаем ответы в 3 колонки
   \begin{choices} % опция [o] не рандомизирует порядок ответов
      \correctchoice{может быть не равна $TSS$}
      \wrongchoice{обязательно равна $TSS$}
      \end{choices}
  \end{multicols}
  \end{questionmult}
}

\element{combat_15}{ % в фигурных скобках название группы вопросов
 \AMCnoCompleteMulti
  \begin{questionmult}{5} % тип вопроса (questionmult --- множественный выбор) и в фигурных --- номер вопроса
  При наличии ошибок измерения зависимой переменной МНК-оценки коэффициентов модели
 \begin{multicols}{2} % располагаем ответы в 3 колонки
   \begin{choices} % опция [o] не рандомизирует порядок ответов
      \correctchoice{состоятельны}
      \wrongchoice{несостоятельны}
      \end{choices}
  \end{multicols}
  \end{questionmult}
}

\element{combat_15}{ % в фигурных скобках название группы вопросов
 \AMCnoCompleteMulti
  \begin{questionmult}{6} % тип вопроса (questionmult --- множественный выбор) и в фигурных --- номер вопроса
  Если выполнены все предпосылки теоремы Гаусса-Маркова, но остатки модели не подчиняются нормальному закону распределения, то МНК-оценки коэффициентов регрессии являются
 \begin{multicols}{2} % располагаем ответы в 3 колонки
   \begin{choices} % опция [o] не рандомизирует порядок ответов
      \correctchoice{несмещёнными}
      \wrongchoice{смещёнными}
      \end{choices}
  \end{multicols}
  \end{questionmult}
}


\element{combat_15}{ % в фигурных скобках название группы вопросов
 \AMCnoCompleteMulti
  \begin{questionmult}{7} % тип вопроса (questionmult --- множественный выбор) и в фигурных --- номер вопроса
  Индексы вздутия дисперсии (VIF) в случае отсутствия мультиколлинеарности лежат в интервале
 \begin{multicols}{2} % располагаем ответы в 3 колонки
   \begin{choices} % опция [o] не рандомизирует порядок ответов
      \correctchoice{$[1;+\infty)$}
      \wrongchoice{$[0;1]$}
      \end{choices}
  \end{multicols}
  \end{questionmult}
}


\element{combat_15}{ % в фигурных скобках название группы вопросов
 \AMCnoCompleteMulti
  \begin{questionmult}{8} % тип вопроса (questionmult --- множественный выбор) и в фигурных --- номер вопроса
  Нулевая гипотеза в тесте Дарбина-Уотсона состоит в
 \begin{multicols}{2} % располагаем ответы в 3 колонки
   \begin{choices} % опция [o] не рандомизирует порядок ответов
      \correctchoice{отсутствии автокорреляции}
      \wrongchoice{наличии автокорреляции}
      \end{choices}
  \end{multicols}
  \end{questionmult}
}

\element{combat_15}{ % в фигурных скобках название группы вопросов
 \AMCnoCompleteMulti
  \begin{questionmult}{9} % тип вопроса (questionmult --- множественный выбор) и в фигурных --- номер вопроса
  Если в модель добавили незначимый фактор, то коэффициент детерминации $R^2$
 \begin{multicols}{3} % располагаем ответы в 3 колонки
   \begin{choices} % опция [o] не рандомизирует порядок ответов
      \correctchoice{вырастет}
      \wrongchoice{упадёт}
      \wrongchoice{не изменится}
      \end{choices}
  \end{multicols}
  \end{questionmult}
}

\element{combat_15}{ % в фигурных скобках название группы вопросов
 \AMCnoCompleteMulti
  \begin{questionmult}{10} % тип вопроса (questionmult --- множественный выбор) и в фигурных --- номер вопроса
  При диагностике автокорреляции третьего порядка тест Бройша-Годфри
 \begin{multicols}{3} % располагаем ответы в 3 колонки
   \begin{choices} % опция [o] не рандомизирует порядок ответов
      \correctchoice{применим}
      \wrongchoice{неприменим}
      \end{choices}
  \end{multicols}
  \end{questionmult}
}


\element{combat_15_v2}{ % в фигурных скобках название группы вопросов
 \AMCnoCompleteMulti
  \begin{questionmult}{11} % тип вопроса (questionmult --- множественный выбор) и в фигурных --- номер вопроса
  Нормальность остатков является одной из предпосылок теоремы Гаусса-Маркова
 \begin{multicols}{3} % располагаем ответы в 3 колонки
   \begin{choices} % опция [o] не рандомизирует порядок ответов
      \correctchoice{неверно}
      \wrongchoice{верно}
      \end{choices}
  \end{multicols}
  \end{questionmult}
}


\element{combat_15_v2}{ % в фигурных скобках название группы вопросов
 \AMCcompleteMulti
  \begin{questionmult}{12} % тип вопроса (questionmult --- множественный выбор) и в фигурных --- номер вопроса
  В случае гетероскедастичности применение стандартных ошибок в форме Уайта позволяет сделать оценки коэффициентов регрессии
 \begin{multicols}{3} % располагаем ответы в 3 колонки
   \begin{choices} % опция [o] не рандомизирует порядок ответов
      \wrongchoice{несмещёнными}
      \wrongchoice{состоятельными}
      \wrongchoice{эффективными}
      \end{choices}
  \end{multicols}
  \end{questionmult}
}

\element{combat_15_v2}{ % в фигурных скобках название группы вопросов
 \AMCnoCompleteMulti
  \begin{questionmult}{13} % тип вопроса (questionmult --- множественный выбор) и в фигурных --- номер вопроса
  С помощью МНК оценивается модель $y_i = \beta_1 + \beta_2 x_i + \varepsilon_i$. Наблюдения представляют собой случайную выборку, и $\Cov(\varepsilon_i, x_i) = 1$. В этом случае $\plim \hat \beta_2^{ols}$
 \begin{multicols}{3} % располагаем ответы в 3 колонки
   \begin{choices} % опция [o] не рандомизирует порядок ответов
      \correctchoice{не равен $\beta_2$}
      \wrongchoice{равен $\beta_2$}
      \end{choices}
  \end{multicols}
  \end{questionmult}
}


\element{combat_15_v2}{ % в фигурных скобках название группы вопросов
 \AMCnoCompleteMulti
  \begin{questionmult}{14} % тип вопроса (questionmult --- множественный выбор) и в фигурных --- номер вопроса
  После применения МНК к модели $y_i=\beta x_i+\varepsilon_i$ сумма остатков $\sum \hat \varepsilon_i$
 \begin{multicols}{3} % располагаем ответы в 3 колонки
   \begin{choices} % опция [o] не рандомизирует порядок ответов
      \correctchoice{не равна нулю}
      \wrongchoice{равна нулю}
      \end{choices}
  \end{multicols}
  \end{questionmult}
}

\element{combat_15_v2}{ % в фигурных скобках название группы вопросов
 \AMCnoCompleteMulti
  \begin{questionmult}{15} % тип вопроса (questionmult --- множественный выбор) и в фигурных --- номер вопроса
  В результате применения МНК к модели $y_i=\beta_1 + \beta_2 x_i+\varepsilon_i$ сумма $\sum x_i \hat \varepsilon_i$
 \begin{multicols}{3} % располагаем ответы в 3 колонки
   \begin{choices} % опция [o] не рандомизирует порядок ответов
      \correctchoice{обязательно равна нулю}
      \wrongchoice{может быть не равна нулю}
      \end{choices}
  \end{multicols}
  \end{questionmult}
}

\element{combat_15_v2}{ % в фигурных скобках название группы вопросов
 \AMCnoCompleteMulti
  \begin{questionmult}{16} % тип вопроса (questionmult --- множественный выбор) и в фигурных --- номер вопроса
  В случае мультиколлинеарности применение гребневой регрессии (ridge-regression) делает оценки коэффициентов
 \begin{multicols}{3} % располагаем ответы в 3 колонки
   \begin{choices} % опция [o] не рандомизирует порядок ответов
      \correctchoice{смещёнными}
      \wrongchoice{несмещёнными}
      \end{choices}
  \end{multicols}
  \end{questionmult}
}

\element{combat_15_v2}{ % в фигурных скобках название группы вопросов
 \AMCnoCompleteMulti
  \begin{questionmult}{17} % тип вопроса (questionmult --- множественный выбор) и в фигурных --- номер вопроса
  В случае мультиколлинеарности оценки дисперсий коэффициентов модели становятся
 \begin{multicols}{3} % располагаем ответы в 3 колонки
   \begin{choices} % опция [o] не рандомизирует порядок ответов
      \correctchoice{несмещёнными}
      \wrongchoice{смещёнными}
      \end{choices}
  \end{multicols}
  \end{questionmult}
}

\element{combat_15_v2}{ % в фигурных скобках название группы вопросов
 \AMCcompleteMulti
  \begin{questionmult}{18} % тип вопроса (questionmult --- множественный выбор) и в фигурных --- номер вопроса
  После применения МНК к исходной модели дополнительно можно оценить модель $\ln(\hat \varepsilon_i^2 )=\gamma_1 + \gamma_2 \ln(x_i) + u_i$  для диагностики
 \begin{multicols}{2} % располагаем ответы в 3 колонки
   \begin{choices} % опция [o] не рандомизирует порядок ответов
      \correctchoice{гетероскедастичности}
      \wrongchoice{автокорреляции}
      \wrongchoice{мультиколлинеарности}
      \end{choices}
  \end{multicols}
  \end{questionmult}
}

\element{combat_15_v2}{ % в фигурных скобках название группы вопросов
 \AMCcompleteMulti
  \begin{questionmult}{19} % тип вопроса (questionmult --- множественный выбор) и в фигурных --- номер вопроса
  Для сравнения качества моделей $y_i=\beta_1+ \beta_2 x_i+\varepsilon_i$ и $\ln(y_i)=\gamma_1+ \gamma_2 x_i + \varepsilon_i$, оцененных на одном наборе данных, используют
 \begin{multicols}{2} % располагаем ответы в 3 колонки
   \begin{choices} % опция [o] не рандомизирует порядок ответов
      \wrongchoice{коэффициент детерминации $R^2$}
      \wrongchoice{скорректированный коэффициент $R_{adj}^2$}
      \end{choices}
  \end{multicols}
  \end{questionmult}
}


\element{combat_15_v2}{ % в фигурных скобках название группы вопросов
 \AMCnoCompleteMulti
  \begin{questionmult}{20} % тип вопроса (questionmult --- множественный выбор) и в фигурных --- номер вопроса
  При диагностике автокорреляции первого порядка тест Бройша-Годфри
 \begin{multicols}{2} % располагаем ответы в 3 колонки
   \begin{choices} % опция [o] не рандомизирует порядок ответов
      \correctchoice{применим}
      \wrongchoice{неприменим}
      \end{choices}
  \end{multicols}
  \end{questionmult}
}




















\element{combat_15_v3}{ % в фигурных скобках название группы вопросов
 \AMCnoCompleteMulti
  \begin{questionmult}{1} % тип вопроса (questionmult --- множественный выбор) и в фигурных --- номер вопроса
  Ошибки измерения независимой переменной являются одной из причин
 \begin{multicols}{2} % располагаем ответы в 3 колонки
   \begin{choices} % опция [o] не рандомизирует порядок ответов
      \correctchoice{эндогенности}
      \wrongchoice{мультиколлинеарности}
      \wrongchoice{автокорреляции}
      \end{choices}
  \end{multicols}
  \end{questionmult}
}

\element{combat_15_v3}{ % в фигурных скобках название группы вопросов
 \AMCnoCompleteMulti
  \begin{questionmult}{2} % тип вопроса (questionmult --- множественный выбор) и в фигурных --- номер вопроса
  Во временных рядах гетероскедастичность
 \begin{multicols}{2} % располагаем ответы в 3 колонки
   \begin{choices} % опция [o] не рандомизирует порядок ответов
      \correctchoice{возможна}
      \wrongchoice{невозможна}
      \end{choices}
  \end{multicols}
  \end{questionmult}
}

\element{combat_15_v3}{ % в фигурных скобках название группы вопросов
 \AMCnoCompleteMulti
  \begin{questionmult}{3} % тип вопроса (questionmult --- множественный выбор) и в фигурных --- номер вопроса
  Оценка $\hb$ называется несмещённой, если с ростом числа наблюдений она стремится к истинной $\beta$
 \begin{multicols}{2} % располагаем ответы в 3 колонки
   \begin{choices} % опция [o] не рандомизирует порядок ответов
      \correctchoice{неверно}
      \wrongchoice{верно}
      \end{choices}
  \end{multicols}
  \end{questionmult}
}


\element{combat_15_v3}{ % в фигурных скобках название группы вопросов
 \AMCnoCompleteMulti
  \begin{questionmult}{4} % тип вопроса (questionmult --- множественный выбор) и в фигурных --- номер вопроса
  С ростом числа наблюдений распределение статистики Дарбина-Уотсона стремится к
 \begin{multicols}{2} % располагаем ответы в 3 колонки
   \begin{choices} % опция [o] не рандомизирует порядок ответов
      \correctchoice{особому распределению Дарбина-Уотсона}
      \wrongchoice{стандартному нормальному распределению}
      \end{choices}
  \end{multicols}
  \end{questionmult}
}

\element{combat_15_v3}{ % в фигурных скобках название группы вопросов
 \AMCcompleteMulti
  \begin{questionmult}{5} % тип вопроса (questionmult --- множественный выбор) и в фигурных --- номер вопроса
  Одним из способов борьбы с нестрогой мультиколлинеарностью является
 %\begin{multicols}{2} % располагаем ответы в 3 колонки
   \begin{choices} % опция [o] не рандомизирует порядок ответов
      \correctchoice{увеличение количества наблюдений}
      \wrongchoice{деление всех регрессоров на одно и то же большое число}
      \wrongchoice{использование взвешенного МНК}
    \end{choices}
 %\end{multicols}
  \end{questionmult}
}

\element{combat_15_v3}{ % в фигурных скобках название группы вопросов
 \AMCnoCompleteMulti
  \begin{questionmult}{6} % тип вопроса (questionmult --- множественный выбор) и в фигурных --- номер вопроса
 Нестрогая мультиколлинеарность нарушает теорему Гаусса-Маркова
 \begin{multicols}{2} % располагаем ответы в 3 колонки
   \begin{choices} % опция [o] не рандомизирует порядок ответов
      \correctchoice{неверно}
      \wrongchoice{верно}
      \end{choices}
  \end{multicols}
  \end{questionmult}
}


\element{combat_15_v3}{ % в фигурных скобках название группы вопросов
 \AMCnoCompleteMulti
  \begin{questionmult}{7} % тип вопроса (questionmult --- множественный выбор) и в фигурных --- номер вопроса
  Если ошибки распределены не нормально, то МНК-оценки коэффициентов регрессии
 \begin{multicols}{2} % располагаем ответы в 3 колонки
   \begin{choices} % опция [o] не рандомизирует порядок ответов
      \correctchoice{эффективны и несмещены}
      \wrongchoice{неэффективны и смещены}
      \wrongchoice{эффективны и смещены}
      \wrongchoice{неэффективны и несмещены}
    \end{choices}
  \end{multicols}
  \end{questionmult}
}


\element{combat_15_v3}{ % в фигурных скобках название группы вопросов
 \AMCnoCompleteMulti
  \begin{questionmult}{8} % тип вопроса (questionmult --- множественный выбор) и в фигурных --- номер вопроса
 Если в модели присутствуют лаги независимой переменной, то тест Дарбина-Уотсона
 \begin{multicols}{2} % располагаем ответы в 3 колонки
   \begin{choices} % опция [o] не рандомизирует порядок ответов
      \correctchoice{применим}
      \wrongchoice{неприменим}
      \end{choices}
  \end{multicols}
  \end{questionmult}
}

\element{combat_15_v3}{ % в фигурных скобках название группы вопросов
 \AMCnoCompleteMulti
  \begin{questionmult}{9} % тип вопроса (questionmult --- множественный выбор) и в фигурных --- номер вопроса
  При гетероскедастичности оценки коэффициентов
 \begin{multicols}{2} % располагаем ответы в 3 колонки
   \begin{choices} % опция [o] не рандомизирует порядок ответов
      \correctchoice{остаются несмещёнными}
      \wrongchoice{в среднем завышены}
      \wrongchoice{в среднем занижены}
      \end{choices}
  \end{multicols}
  \end{questionmult}
}

\element{combat_15_v3}{ % в фигурных скобках название группы вопросов
 \AMCnoCompleteMulti
  \begin{questionmult}{10} % тип вопроса (questionmult --- множественный выбор) и в фигурных --- номер вопроса
  У разностного  уравнения $y_t=0.1y_{t-1} + \e_t +0.4\e_{t-1}$
 \begin{multicols}{3} % располагаем ответы в 3 колонки
   \begin{choices} % опция [o] не рандомизирует порядок ответов
      \correctchoice{есть единственное стационарное решение}
      \wrongchoice{нет стационарных решений}
      \wrongchoice{есть бесконечное количество стационарных решений}
    \end{choices}
  \end{multicols}
  \end{questionmult}
}























\element{combat_15_v4}{ % в фигурных скобках название группы вопросов
 \AMCcompleteMulti
  \begin{questionmult}{1} % тип вопроса (questionmult --- множественный выбор) и в фигурных --- номер вопроса
  В теореме Гаусса-Маркова предполагается, что ошибки
 \begin{multicols}{2} % располагаем ответы в 3 колонки
   \begin{choices} % опция [o] не рандомизирует порядок ответов
      \wrongchoice{имеют нулевое математическое ожидание и единичную дисперсию}
      \wrongchoice{имеют ненулевое математическое ожидание и неединичную дисперсию}
    \end{choices}
  \end{multicols}
  \end{questionmult}
}

\element{combat_15_v4}{ % в фигурных скобках название группы вопросов
 \AMCnoCompleteMulti
  \begin{questionmult}{2} % тип вопроса (questionmult --- множественный выбор) и в фигурных --- номер вопроса
  При условной гетероскедастичности оценки коэффициентов
 \begin{multicols}{2} % располагаем ответы в 3 колонки
   \begin{choices} % опция [o] не рандомизирует порядок ответов
      \correctchoice{состоятельны}
      \wrongchoice{несостоятельны}
      \end{choices}
  \end{multicols}
  \end{questionmult}
}

\element{combat_15_v4}{ % в фигурных скобках название группы вопросов
 \AMCcompleteMulti
  \begin{questionmult}{3} % тип вопроса (questionmult --- множественный выбор) и в фигурных --- номер вопроса
  Двухшаговый метод наименьших квадратов --- это стандартный способ борьбы с
 \begin{multicols}{2} % располагаем ответы в 3 колонки
   \begin{choices} % опция [o] не рандомизирует порядок ответов
      \correctchoice{эндогенностью}
      \wrongchoice{автокорреляцией}
      \wrongchoice{гетероскедастичностью}
      \end{choices}
  \end{multicols}
  \end{questionmult}
}


\element{combat_15_v4}{ % в фигурных скобках название группы вопросов
 \AMCnoCompleteMulti
  \begin{questionmult}{4} % тип вопроса (questionmult --- множественный выбор) и в фигурных --- номер вопроса
  После применения МНК к модели $y_i=\beta x_i + \varepsilon_i$   сумма $\sum \he_i^2$
 \begin{multicols}{2} % располагаем ответы в 3 колонки
   \begin{choices} % опция [o] не рандомизирует порядок ответов
      \correctchoice{равна нулю}
      \wrongchoice{не обязательно равна нулю}
      \end{choices}
  \end{multicols}
  \end{questionmult}
}

\element{combat_15_v4}{ % в фигурных скобках название группы вопросов
 \AMCnoCompleteMulti
  \begin{questionmult}{5} % тип вопроса (questionmult --- множественный выбор) и в фигурных --- номер вопроса
  При автокорреляции МНК-оценки коэффициентов являются
 \begin{multicols}{2} % располагаем ответы в 3 колонки
   \begin{choices} % опция [o] не рандомизирует порядок ответов
      \correctchoice{несмещёнными}
      \wrongchoice{смещёнными}
      \end{choices}
  \end{multicols}
  \end{questionmult}
}

\element{combat_15_v4}{ % в фигурных скобках название группы вопросов
 \AMCnoCompleteMulti
  \begin{questionmult}{6} % тип вопроса (questionmult --- множественный выбор) и в фигурных --- номер вопроса
  Нестрогая мультиколлинеарность --- это одно из нарушений предпосылок теоремы Гаусса-Маркова
 \begin{multicols}{2} % располагаем ответы в 3 колонки
   \begin{choices} % опция [o] не рандомизирует порядок ответов
      \correctchoice{неверно}
      \wrongchoice{верно}
      \end{choices}
  \end{multicols}
  \end{questionmult}
}


\element{combat_15_v4}{ % в фигурных скобках название группы вопросов
 \AMCnoCompleteMulti
  \begin{questionmult}{7} % тип вопроса (questionmult --- множественный выбор) и в фигурных --- номер вопроса
  В случае мультиколлинеарности оценки дисперсий коэффициентов являются
 \begin{multicols}{3} % располагаем ответы в 3 колонки
   \begin{choices} % опция [o] не рандомизирует порядок ответов
      \correctchoice{несмещёнными}
      \wrongchoice{в среднем завышенными}
      \wrongchoice{в среднем заниженными}
      \end{choices}
  \end{multicols}
  \end{questionmult}
}


\element{combat_15_v4}{ % в фигурных скобках название группы вопросов
 \AMCnoCompleteMulti
  \begin{questionmult}{8} % тип вопроса (questionmult --- множественный выбор) и в фигурных --- номер вопроса
  Тест Дарбина-Уотсона в регрессии без свободного члена
 \begin{multicols}{2} % располагаем ответы в 3 колонки
   \begin{choices} % опция [o] не рандомизирует порядок ответов
      \correctchoice{неприменим}
      \wrongchoice{применим}
      \end{choices}
  \end{multicols}
  \end{questionmult}
}

\element{combat_15_v4}{ % в фигурных скобках название группы вопросов
 \AMCnoCompleteMulti
  \begin{questionmult}{9} % тип вопроса (questionmult --- множественный выбор) и в фигурных --- номер вопроса
  Нулевой гипотезой в тесте Уайта является гипотеза о
 \begin{multicols}{2} % располагаем ответы в 3 колонки
   \begin{choices} % опция [o] не рандомизирует порядок ответов
      \correctchoice{гомоскедастичности}
      \wrongchoice{гетероскедастичности}
      \wrongchoice{наличии автокорреляции}
      \wrongchoice{отсутствии автокорреляции}
    \end{choices}
  \end{multicols}
  \end{questionmult}
}

\element{combat_15_v4}{ % в фигурных скобках название группы вопросов
 \AMCnoCompleteMulti
  \begin{questionmult}{10} % тип вопроса (questionmult --- множественный выбор) и в фигурных --- номер вопроса
   Величина $R^2_{adj}$ показывает, какую долю дисперсии зависимой переменной объясняют использованные регрессоры
 \begin{multicols}{3} % располагаем ответы в 3 колонки
   \begin{choices} % опция [o] не рандомизирует порядок ответов
      \correctchoice{неверно}
      \wrongchoice{верно}
      \end{choices}
  \end{multicols}
  \end{questionmult}
}


\section*{Часть 1. Тест.}

\onecopy{1}{

\cleargroup{combat}
\copygroup[10]{combat_kr_01_16_retake}{combat}
%\shufflegroup{combat}
\insertgroup{combat}

}

\section*{Часть 2. Задачи.}


\begin{enumerate}

\item Эконометресса Агриппина изучает связь числа колючек на кактусе $spines_i$, (в штуках) от его высоты, $height_i$, (в годах):
\[
milk_i = \beta_1 + \beta_2 age_i + u_i
\]

\begin{tabular}{lr} \toprule
Показатель & Значение \\
\midrule
$RSS$                        & \textbf{240} \\
$ESS$                        & \textbf{260} \\
$TSS$                        & \textbf{B1} \\
$R^2$                        & \textbf{B2} \\
Стандартная ошибка регрессии & \textbf{B3} \\
Количество наблюдений        & 72 \\
\bottomrule
\end{tabular}

\begin{tabular}{lrrrrrr} \toprule
Коэффициент & Оценка & $se(\hb)$ & t-статистика & P-значение & Левая (95\%) & Правая (95\%) \\
\midrule
Константа & 6.391 & 1.007 & \textbf{В4} & 0.000 & \textbf{В6} & \textbf{В7} \\
$height$ & \textbf{В8} & \textbf{В9} & \textbf{В10} & 0.000 & 0.036 & 0.119 \\
\bottomrule
\end{tabular}

Найдите пропущенные числа \textbf{B1}--\textbf{B10}.




\item Агриппина решила изучить как на количество цветков на кактусе, $Y_i$, влияет количество иголок, $X_i$. После замеров по 5 кактусам, она получила следующие данные $Y = (0,1,0,4,0)$, $X = (3,4,3,6,4)$. Агриппина предполагает корректность линейной модели $Y_i = \beta_1 + \beta_2 X_i + u_i$.
\begin{enumerate}
\item Найдите МНК-оценки коэффициентов регресси
\item Найдите $RSS$, $ESS$, $TSS$ и $R^2$
\end{enumerate}



\item Для модели $Y_i = \beta_1 + \beta_2 X_i + u_i$ выполнены все предпосылки теоремы Гаусса-Маркова, а случайные ошибки нормально распределены. Известны все значения остатков $e_i$, и часть значений $X_i$, $Y_i$, $\hat Y_i$.

\begin{tabular}{lllll}
\toprule
$X_i$       & 3 & 4 & . & . \\
$Y_i$       & 6 & . & . & 6 \\
$\hat Y_i$ & . & 7 & 4 & . \\
$e_i$       & -2 & 2 & 1 & -1 \\
\bottomrule
\end{tabular}

\begin{enumerate}
\item Восстановите пропуски
\item Найдите МНК-оценки коэффициентов регрессии
\item Найдите стандартную ошибку коэффициента $\hat \beta_2$
\item Постройте 95\%-ый доверительный интервал для коэффициента $\hat \beta_2$
\item Проверьте гипотезу о незначимости коэффициента $\beta_2$ на уровне значимости 5\%
\end{enumerate}


\item Дайте определение величинам $RSS$, $ESS$, $TSS$. Аккуратно сформулируйте теорему об их взаимосвязи.

\item Докажите, что коэффициенты $R^2$ в парных регрессиях $Y_i = \alpha_1 + \alpha_2 X_i + u_i$ и $X_i = \beta_1 + \beta_2 Y_i + v_i$ совпадают.

\end{enumerate}

\end{document}
