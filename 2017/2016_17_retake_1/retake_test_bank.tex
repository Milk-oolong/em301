\element{2017_fall_retake_1}{ % в фигурных скобках название группы вопросов
%  %\AMCnoCompleteMulti
\begin{questionmult}{1} % тип вопроса (questionmult — множественный выбор) и в фигурных — номер вопроса
Если основная гипотеза в тесте Дики-Фуллера не отвергается, то временной ряд является
\begin{multicols}{2} % располагаем ответы в 3 колонки
\begin{choices} % опция [o] не рандомизирует порядок ответов
       \correctchoice{нестационарным}
       \wrongchoice{стационарным}
       \wrongchoice{коинтегрированным}
       \wrongchoice{стационарным в первых разностях}
       \wrongchoice{нормально распределённым}
    \end{choices}
   \end{multicols}
\end{questionmult}
}


\element{2017_fall_retake_1}{ % в фигурных скобках название группы вопросов
%  %\AMCnoCompleteMulti
\begin{questionmult}{2} % тип вопроса (questionmult — множественный выбор) и в фигурных — номер вопроса
По результатам теста Дарбина-Уотсона НЕВОЗМОЖНО сделать вывод о
\begin{multicols}{2} % располагаем ответы в 3 колонки
\begin{choices} % опция [o] не рандомизирует порядок ответов
       \correctchoice{наличии коинтеграции}
       \wrongchoice{наличии положительной автокорреляции}
       \wrongchoice{наличии отрицательной автокорреляции}
       \wrongchoice{отсутствии автокорреляции}
       \wrongchoice{неопределенности в отношении автокорреляции}
    \end{choices}
   \end{multicols}
\end{questionmult}
}


\element{2017_fall_retake_1}{ % в фигурных скобках название группы вопросов
%  %\AMCnoCompleteMulti
\begin{questionmult}{3} % тип вопроса (questionmult — множественный выбор) и в фигурных — номер вопроса
Ряд с детерминированным линейным трендом $Y_t = \beta_0 + \beta_1 t + \epsilon_t$
\begin{multicols}{2} % располагаем ответы в 3 колонки
\begin{choices} % опция [o] не рандомизирует порядок ответов
       \correctchoice{является нестационарным, потому что $\E(Y_t)$ зависит от времени}
       \wrongchoice{является нестационарным, потому что $\Var(Y_t)$ зависит от времени}
       \wrongchoice{является нестационарным, потому что $\Cov(Y_t, Y_{t-1})$ зависит от времени}
       \wrongchoice{является стационарным}
       \wrongchoice{является нестационарным в первых разностях}
    \end{choices}
   \end{multicols}
\end{questionmult}
}


% bad style
\element{2017_fall_retake_1}{ % в фигурных скобках название группы вопросов
%  %\AMCnoCompleteMulti
\begin{questionmult}{4} % тип вопроса (questionmult — множественный выбор) и в фигурных — номер вопроса
Статистика $T \approx 2(1-\hat\rho)$, где $\hat\rho$ - оценка коэффициента автокорреляции, используется в
\begin{multicols}{3} % располагаем ответы в 3 колонки
\begin{choices} % опция [o] не рандомизирует порядок ответов
       \correctchoice{тесте Дарбина-Уотсона}
       \wrongchoice{h-тесте Дарбина}
       \wrongchoice{тесте Бройша-Годфри}
       \wrongchoice{тесте Льюнга-Бокса}
       \wrongchoice{PE-тесте МакКиннона}
    \end{choices}
\end{multicols}
\end{questionmult}
}







\element{2017_fall_retake_1}{ % в фигурных скобках название группы вопросов
%  %\AMCnoCompleteMulti
\begin{questionmult}{5} % тип вопроса (questionmult — множественный выбор) и в фигурных — номер вопроса
Высокая по модулю корреляция между $e_t$ и $e_{t-1}$ может говорить о
\begin{multicols}{2} % располагаем ответы в 3 колонки
\begin{choices} % опция [o] не рандомизирует порядок ответов
       \correctchoice{автокорреляции}
       \wrongchoice{гетероскедастичности}
       \wrongchoice{нестационарности}
       \wrongchoice{мультиколлинеарности}
       \wrongchoice{коинтеграции}
    \end{choices}
   \end{multicols}
\end{questionmult}
}

\element{2017_fall_retake_1}{ % в фигурных скобках название группы вопросов
%  %\AMCnoCompleteMulti
\begin{questionmult}{6} % тип вопроса (questionmult — множественный выбор) и в фигурных — номер вопроса
Высокая по модулю корреляция между $e_t^2$ и $X_t$ может говорить о
\begin{multicols}{2} % располагаем ответы в 3 колонки
\begin{choices} % опция [o] не рандомизирует порядок ответов
       \correctchoice{гетероскедастичности}
       \wrongchoice{автокорреляции}
       \wrongchoice{нестационарности}
       \wrongchoice{мультиколлинеарности}
       \wrongchoice{коинтеграции}
    \end{choices}
   \end{multicols}
\end{questionmult}
}

\element{2017_fall_retake_1}{ % в фигурных скобках название группы вопросов
% \AMCnoCompleteMulti
\begin{questionmult}{7} % тип вопроса (questionmult — множественный выбор) и в фигурных — номер вопроса
Высокая по модулю корреляция между $Y_t$ и $X_t$ может говорить о
\begin{multicols}{2} % располагаем ответы в 3 колонки
\begin{choices} % опция [o] не рандомизирует порядок ответов
       %\correctchoice{нет верного ответа}
       \wrongchoice{гетероскедастичности}
       \wrongchoice{автокорреляции}
       \wrongchoice{нестационарности}
       \wrongchoice{мультиколлинеарности}
       \wrongchoice{коинтеграции}
    \end{choices}
   \end{multicols}
\end{questionmult}
}






\element{2017_fall_retake_1}{ % в фигурных скобках название группы вопросов
%  %\AMCnoCompleteMulti
\begin{questionmult}{8} % тип вопроса (questionmult — множественный выбор) и в фигурных — номер вопроса
Выберите верное утверждение о моделях бинарного выбора
\begin{multicols}{3} % располагаем ответы в 3 колонки
\begin{choices} % опция [o] не рандомизирует порядок ответов
       \correctchoice{Предсказанная вероятность в модели логистической регрессии всегда лежит в границах $[0,1]$}
       \wrongchoice{Предсказанная вероятность в линейной вероятностной модели всегда лежит в границах $[0,1]$}
       \wrongchoice{Дамми-переменные в качестве независимых могут использоваться только в логистической модели}
       \wrongchoice{Выбрать между логистической регрессией и линейной вероятностной моделью можно с помощью теста Бокса-Кокса}
       \wrongchoice{В линейной вероятностной модели ошибки распределены нормально}
    \end{choices}
\end{multicols}
\end{questionmult}
}







\element{2017_fall_retake_1}{ % в фигурных скобках название группы вопросов
 %\AMCnoCompleteMulti
  \begin{questionmult}{9} % тип вопроса (questionmult --- множественный выбор) и в фигурных --- номер вопроса
Если квадраты остатков оценённой с помощью МНК регрессионной модели линейно и значимо зависят от регрессора $Z$, то гетероскедастичность можно попытаться устранить,
 \begin{multicols}{2} % располагаем ответы в 3 колонки
   \begin{choices} % опция [o] не рандомизирует порядок ответов
      \correctchoice{поделив исходное уравнение на $\sqrt Z$}
      \wrongchoice{умножив исходное уравнение на $Z$}
      \wrongchoice{поделив исходное уравнение на $Z$}
      \wrongchoice{умножив исходное уравнение на $\sqrt Z$}
      \wrongchoice{поделив исходное уравнение на ${Z^2}$}
      \wrongchoice{умножив исходное уравнение на ${Z^2}$}
   \end{choices}
  \end{multicols}
  \end{questionmult}
}




\element{2017_fall_retake_1}{ % в фигурных скобках название группы вопросов
 %\AMCnoCompleteMulti
  \begin{questionmult}{10} % тип вопроса (questionmult --- множественный выбор) и в фигурных --- номер вопроса
Методом максимального правдоподобия Гоша оценил модель
    \[
    Y_i = \beta_1 + \beta_2 X_{i2} + \beta _3 X_{i3} + \varepsilon_i,
    \]
 где $\e \sim \cN (0,\sigma_{\e}^2I)$, по 9 наблюдениям. Оказалось, что $RSS = 72$.  Оценка дисперсии случайной составляющей равна
 \begin{multicols}{3} % располагаем ответы в 3 колонки
   \begin{choices} % опция [o] не рандомизирует порядок ответов
      \correctchoice{$8$}
      \wrongchoice{$9$}
      \wrongchoice{$10$}
      \wrongchoice{$\sqrt{8}$}
      \wrongchoice{$\sqrt{9}$}
      \wrongchoice{$3$}
   \end{choices}
  \end{multicols}
  \end{questionmult}
}





\element{2017_fall_retake_1}{ % в фигурных скобках название группы вопросов
 %\AMCnoCompleteMulti
  \begin{questionmult}{11} % тип вопроса (questionmult --- множественный выбор) и в фигурных --- номер вопроса
При высоких (больше $10$) значениях $VIF$
 %\begin{multicols}{2} % располагаем ответы в 3 колонки
   \begin{choices} % опция [o] не рандомизирует порядок ответов
      \correctchoice{МНК-оценки коэффициентов регрессии остаются BLUE}
      \wrongchoice{МНК-оценки коэффициентов регрессии становятся неэффективными}
      \wrongchoice{МНК-оценки коэффициентов регрессии становятся несостоятельными}
      \wrongchoice{МНК-оценки коэффициентов регрессии невозможно найти}
      \wrongchoice{отвергается гипотеза о наличии мультиколлинеарности}
      \wrongchoice{необходимо выкинуть из модели часть регрессоров}
   \end{choices}
  %\end{multicols}
  \end{questionmult}
}



\element{2017_fall_retake_1}{ % в фигурных скобках название группы вопросов
 %\AMCnoCompleteMulti
  \begin{questionmult}{12} % тип вопроса (questionmult --- множественный выбор) и в фигурных --- номер вопроса
  Имеются данные по 100 работникам: затраты на проезд в общественном транспорте ($E_i$, руб.), количество часов работы в день ($WH_i$, руб.), количество часов отдыха в день ($LH_i$, руб.) и количество часов сна в день ($SH_i$, руб.). Изветсно, что не всё время работников делится между работой, сном и отдыхом. Известны выборочные корреляции между переменными, $sCorr(WH, LH) = -0.9$, $sCorr(LH, SH) = 0.6$. Была оценена модель:
\[
E_i = {\beta_1} + {\beta_2}WH_i + {\beta _3}LH_i + {\beta _4}SH_i + u_i
\]
Максимальное значение VIF будет не меньше, чем
   \begin{choices} % опция [o] не рандомизирует порядок ответов
      \correctchoice{$\frac{1}{0.19}$}
      \wrongchoice{$\frac{1}{0.81}$}
      \wrongchoice{$\frac{-1}{0.19}$}
      \wrongchoice{$\frac{1}{0.91}$}
      \wrongchoice{$\frac{1}{0.3}$}
   \end{choices}
  \end{questionmult}
}








\element{2017_fall_retake_1}{ % в фигурных скобках название группы вопросов
 %\AMCnoCompleteMulti
  \begin{questionmult}{13} % тип вопроса (questionmult --- множественный выбор) и в фигурных --- номер вопроса
  При нарушении предпосылки теоремы Гаусса-Маркова $\Var(u) = \sigma^2 I$ эффективные и состоятельные оценки коэффициентов можно получить при помощи
\begin{multicols}{2} % располагаем ответы в 3 колонки
   \begin{choices} % опция [o] не рандомизирует порядок ответов
      \correctchoice{обобщенного МНК}
      \wrongchoice{МНК}
      \wrongchoice{робастных в форме Уайта ошибок}
      \wrongchoice{робастных в форме Ньюи-Уэста ошибок}
      \wrongchoice{взятия первых разностей данных}
   \end{choices}
 \end{multicols}
  \end{questionmult}
}



\element{2017_fall_retake_1}{ % в фигурных скобках название группы вопросов
 %\AMCnoCompleteMulti
  \begin{questionmult}{14} % тип вопроса (questionmult --- множественный выбор) и в фигурных --- номер вопроса
  При работе с преобразованными для учёта автокорреляции данными, поправка Прайса-Винстона
   \begin{choices} % опция [o] не рандомизирует порядок ответов
      \correctchoice{необходима для корректного учёта первого наблюдения}
      \wrongchoice{вводится в вектор оценок коэффициентов и делает их состоятельными}
      \wrongchoice{необходима для получения состоятельных оценок коэффициентов}
      \wrongchoice{аналогична использованию робастных стандартных ошибок}
      \wrongchoice{не используется, т.к. нужна для борьбы с гетероскедастичностью}
   \end{choices}
  \end{questionmult}
}






\element{2017_fall_retake_1}{ % в фигурных скобках название группы вопросов
 %\AMCnoCompleteMulti
  \begin{questionmult}{15} % тип вопроса (questionmult --- множественный выбор) и в фигурных --- номер вопроса
К эндогенности НЕ приводит
   \begin{choices} % опция [o] не рандомизирует порядок ответов
      \correctchoice{пропуск переменной, влияющей на $Y$, но не связанной ни с одним из регрессоров}
      \wrongchoice{пропуск переменной, влияющей на $Y$ и связанной с одним из регрессоров}
      \wrongchoice{ошибка в измерении регрессора}
      \wrongchoice{автокорреляция в ошибках при использовании лага зависимой переменной в качестве регрессора}
      \wrongchoice{автокорреляция в ошибках при НЕиспользовании лага зависимой переменной в качестве регрессора}
   \end{choices}
  \end{questionmult}
}


\element{2017_fall_retake_1}{ % в фигурных скобках название группы вопросов
 %\AMCnoCompleteMulti
  \begin{questionmult}{16} % тип вопроса (questionmult --- множественный выбор) и в фигурных --- номер вопроса
  Оценка метода инструментальных переменных имеет вид
   \begin{choices} % опция [o] не рандомизирует порядок ответов
      \correctchoice{$(Z'X)^{-1}Z'Y$}
      \wrongchoice{$(Z'Z)^{-1}Z'Y$}
      \wrongchoice{$Z(Z'Z)^{-1}Z'X$}
      \wrongchoice{$(X'Z(Z'Z)^{-1}Z'X)^{-1}Z'Z(Z'Z)^{-1}X'Y$}
      \wrongchoice{$(X'X)^{-1}X'Y$}
   \end{choices}
  \end{questionmult}
}






\element{2017_fall_retake_1}{ % в фигурных скобках название группы вопросов
 %\AMCnoCompleteMulti
  \begin{questionmult}{17} % тип вопроса (questionmult --- множественный выбор) и в фигурных --- номер вопроса
  Рассмотрим теста Хаусмана для выбора между МНК-оценками и оценками метода инструментальных переменных. Нулевая гипотеза заключается в том, что
   \begin{choices} % опция [o] не рандомизирует порядок ответов
      \correctchoice{обе оценки состоятельны}
      \wrongchoice{МНК оценка не состоятельна, оценка инструментальных переменных - состоятельна}
      \wrongchoice{МНК оценка состоятельна, оценка инструментальных переменных - не состоятельна}
      \wrongchoice{обе оценки эффективны}
      \wrongchoice{эффективна только оценка метода инструментальных переменных}
   \end{choices}
  \end{questionmult}
}


\element{2017_fall_retake_1}{ % в фигурных скобках название группы вопросов
 %\AMCnoCompleteMulti
  \begin{questionmult}{18} % тип вопроса (questionmult --- множественный выбор) и в фигурных --- номер вопроса
  Тест Хаусмана можно использовать для
   \begin{choices} % опция [o] не рандомизирует порядок ответов
      \correctchoice{выбора между моделью со случайными эффектами и моделью с постоянными эффектами}
      \wrongchoice{проверки стационарности временного ряда}
      \wrongchoice{проверки автокорреляции}
      \wrongchoice{проверки наличия панельной структуры в данных}
      \wrongchoice{выбора между моделью в уровнях и моделью в разностях}
   \end{choices}
  \end{questionmult}
}


\element{2017_fall_retake_1}{ % в фигурных скобках название группы вопросов
 %\AMCnoCompleteMulti
  \begin{questionmult}{19} % тип вопроса (questionmult --- множественный выбор) и в фигурных --- номер вопроса
  Если оценить FE и RE-модель на одном наборе данных, то окажется, что
   \begin{choices} % опция [o] не рандомизирует порядок ответов
      \wrongchoice{оценки коэффициентов FE-модели больше оценок RE-модели}
      \wrongchoice{оценки коэффициентов RE-модели больше оценок FE-модели}
      \wrongchoice{оценки коэффициентов совпадают, различаются оценки ковариационной матрицы}
      \wrongchoice{оценки ковариационной матрицы совпадают, оценки коэффициентов различаются}
      \wrongchoice{t-статистики FE и RE моделей совпадают}
   \end{choices}
  \end{questionmult}
}






\element{2017_fall_retake_1}{ % в фигурных скобках название группы вопросов
%  %\AMCnoCompleteMulti
\begin{questionmult}{20} % тип вопроса (questionmult — множественный выбор) и в фигурных — номер вопроса
Оценка регрессионной зависимости с помощью МНК по 4321 наблюдениям имеет вид $\hat Y_i = 7 + 2X_i + 6Z_i$. Оценка ковариационной матрицы имеет вид
\[
\Var(\hb)=
\begin{pmatrix}
1 & 0.1 & 0.2 \\
0.1 & 4 & 1.5 \\
0.2 & 1.5 & 14 \\
\end{pmatrix}.
\]

Середина 95\%-го доверительного интервала для $2 \beta_2 + \beta_3$ примерно равна

\begin{multicols}{3} % располагаем ответы в 3 колонки
\begin{choices} % опция [o] не рандомизирует порядок ответов
       \correctchoice{$24$}
       \wrongchoice{$20$}
       \wrongchoice{$10$}
       \wrongchoice{$36$}
       \wrongchoice{$8$}
    \end{choices}
   \end{multicols}
\end{questionmult}
}
