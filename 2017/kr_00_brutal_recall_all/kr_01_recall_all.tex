\documentclass[12pt, a4paper]{article}

\usepackage[top=1.5cm, left=2cm, right=2cm, bottom=1.5cm]{geometry} % размер текста на странице

\usepackage{tikz} % картинки в tikz
\usepackage{microtype} % свешивание пунктуации

\usepackage{array} % для столбцов фиксированной ширины

\usepackage{indentfirst} % отступ в первом параграфе

\usepackage{sectsty} % для центрирования названий частей
\allsectionsfont{\centering}

\usepackage{amsmath} % куча стандартных математических плюшек
\usepackage{amssymb} % и символов

\usepackage{multicol} % текст в несколько колонок

\usepackage{lastpage} % чтобы узнать номер последней страницы

\usepackage{enumitem} % дополнительные плюшки для списков
%  например \begin{enumerate}[resume] позволяет продолжить нумерацию в новом списке




\usepackage{fontspec} % хз
\usepackage{polyglossia} % для выбора языка в xelatex

\setmainlanguage{russian}
\setotherlanguages{english}

% download "Linux Libertine" fonts:
% http://www.linuxlibertine.org/index.php?id=91&L=1
\setmainfont{Linux Libertine O} % or Helvetica, Arial, Cambria
% why do we need \newfontfamily:
% http://tex.stackexchange.com/questions/91507/
\newfontfamily{\cyrillicfonttt}{Linux Libertine O}

\AddEnumerateCounter{\asbuk}{\russian@alph}{щ} % для списков с русскими буквами
\setlist[enumerate, 2]{label=\asbuk*),ref=\asbuk*} % списки уровня 2 будут буквами а) б) ...

\usepackage{todonotes} % для вставки в документ заметок о том, что осталось сделать
% \todo[inline]{Здесь надо коэффициенты исправить}
% \missingfigure{Здесь будет картина Последний день Помпеи}
% команда \listoftodos — печатает все поставленные \todo'шки

\usepackage{booktabs} % красивые таблицы
% заповеди из документации:
% 1. Не используйте вертикальные линии
% 2. Не используйте двойные линии
% 3. Единицы измерения помещайте в шапку таблицы
% 4. Не сокращайте .1 вместо 0.1
% 5. Повторяющееся значение повторяйте, а не говорите "то же"


% \usepackage[left=1cm,right=1cm,top=1cm,bottom=1cm]{geometry}

\usepackage{fancyhdr} % весёлые колонтитулы
\pagestyle{fancy}
\lhead{Эконометрика, праздник номер 1!}
\chead{Вспомнить всё!}
\rhead{11.09.2017}
\lfoot{}
\cfoot{}
\rfoot{\thepage/\pageref{LastPage}}
\renewcommand{\headrulewidth}{0.4pt}
\renewcommand{\footrulewidth}{0.4pt}

\DeclareMathOperator{\tr}{tr}
\DeclareMathOperator{\E}{\mathbb{E}}
\let\P\relax
\DeclareMathOperator{\P}{\mathbb{P}}
\DeclareMathOperator{\Var}{\mathbb{V}ar}
\DeclareMathOperator{\Cov}{\mathbb{C}ov}

\begin{document}


\begin{enumerate}
\item Найдите длины векторов $a=(1,1,1)$ и $b=(1,4,6)$ и косинус угла между ними. Найдите длину проекции вектора $b$ на вектор $a$.

\item Сформулируйте теорему Фалеса. Сформулируйте и докажите теорему Пифагора.

%\item На плоскости $\alpha$ лежит прямая $\ell$. Вне плоскости $\alpha$ лежит точка $C$. Ромео проецирует точку $C$ на прямую $\ell$ и получает точку $R$. Джульетта проецирует точку $C$ сначала на плоскость $\alpha$, а затем проецирует полученную точку $A$ на прямую $\ell$. После двух действий Джульетта получает точку $D$. Обязательно ли $R$ и $D$ совпадают?

\item Для матрицы

\[
A=\begin{pmatrix}
6 & 5 \\
5 & 6 \\
\end{pmatrix}
\]

\begin{enumerate}
\item Найдите собственные числа и собственные векторы матрицы
\item Найдите $\det (A)$, $\tr(A)$
\item Найдите собственные числа матрицы $A^{2017}$, $\det (A^{2017})$ и $\tr(A^{2017})$
\end{enumerate}

%\item Известно, что $X$ — матрица размера $n \times k$ и $n>k$, известно, что $X'X$ обратима. Рассмотрим матрицу $H=X(X'X)^{-1}X'$. Укажите размер матрицы $H$, найдите $H^{2016}$, $\tr(H)$, $\det(H)$, собственные числа матрицы $H$. Штрих означает транспонирование.

\item Занудная халява: известно, что $\Cov(X, Y)=5$, $\Var(X)=16$, $\Var(Y)=25$, $\E(X)=10$, $\E(Y)=-5$. Найдите $\Cov(X+2Y, Y-X)$, $\Var(X+2Y)$, $\E(X+2Y)$.

\item Блондинка Маша 100 раз выходила на улицу и при этом 40 раз встретила динозавра. Постройте 95\% доверительный интервал для вероятности встретить динозавра. На уровне 5\% проверьте гипотезу о том, что данная вероятность равна $0.5$ против альтернативной гипотезы об отличии данной вероятности от $0.5$.

\item В кошельке 5 монеток, три золотых и две серебряных. Маша берёт наугад две монетки по очереди. Маше достались одинаковые монетки. Какова условная вероятность того, что обе золотые?
\end{enumerate}


\vspace{10pt}

\begin{enumerate}
\item Найдите длины векторов $a=(1,1,1)$ и $b=(1,4,6)$ и косинус угла между ними. Найдите длину проекции вектора $b$ на вектор $a$.

\item Сформулируйте теорему Фалеса. Сформулируйте и докажите теорему Пифагора.

%\item На плоскости $\alpha$ лежит прямая $\ell$. Вне плоскости $\alpha$ лежит точка $C$. Ромео проецирует точку $C$ на прямую $\ell$ и получает точку $R$. Джульетта проецирует точку $C$ сначала на плоскость $\alpha$, а затем проецирует полученную точку $A$ на прямую $\ell$. После двух действий Джульетта получает точку $D$. Обязательно ли $R$ и $D$ совпадают?

\item Для матрицы

\[
A=\begin{pmatrix}
6 & 5 \\
5 & 6 \\
\end{pmatrix}
\]

\begin{enumerate}
\item Найдите собственные числа и собственные векторы матрицы
\item Найдите $\det (A)$, $\tr(A)$
\item Найдите собственные числа матрицы $A^{2017}$, $\det (A^{2017})$ и $\tr(A^{2017})$
\end{enumerate}

%\item Известно, что $X$ — матрица размера $n \times k$ и $n>k$, известно, что $X'X$ обратима. Рассмотрим матрицу $H=X(X'X)^{-1}X'$. Укажите размер матрицы $H$, найдите $H^{2016}$, $\tr(H)$, $\det(H)$, собственные числа матрицы $H$. Штрих означает транспонирование.

\item Занудная халява: известно, что $\Cov(X, Y)=5$, $\Var(X)=16$, $\Var(Y)=25$, $\E(X)=10$, $\E(Y)=-5$. Найдите $\Cov(X+2Y, Y-X)$, $\Var(X+2Y)$, $\E(X+2Y)$.

\item Блондинка Маша 100 раз выходила на улицу и при этом 40 раз встретила динозавра. Постройте 95\% доверительный интервал для вероятности встретить динозавра. На уровне 5\% проверьте гипотезу о том, что данная вероятность равна $0.5$ против альтернативной гипотезы об отличии данной вероятности от $0.5$.

\item В кошельке 5 монеток, три золотых и две серебряных. Маша берёт наугад две монетки по очереди. Маше достались одинаковые монетки. Какова условная вероятность того, что обе золотые?
\end{enumerate}








\end{document}
